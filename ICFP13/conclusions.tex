\section{Related Work}

\label{sect:related}

The work presented in this paper arose from dissatisfaction with the 
lack of composability of monad transformers, and a belief that a dependently
typed language ought to handle side-effects more flexibly, in such a way
that it would be possible to reason about side-effecting programs. The
inspiration for using an \remph{algebraic} representation of effects was
Bauer and Pretnar's \remph{Eff} language~\cite{Bauer}, a language
based around handlers of algebraic effects. In our \Eff{} system, we have
found type classes to be a natural way of implementing effect handlers,
particular because they allow different effect handlers to be run in
different contexts. Other languages aim to bring effects into their
type system, such as Disciple~\cite{Lippmeier2009}, 
Frank\footnote{\url{https://personal.cis.strath.ac.uk/conor.mcbride/pub/Frank/}}
and
Koka\footnote{\url{http://research.microsoft.com/en-us/projects/koka/}}.
These languages are built on well-studied theoretical
foundations~\cite{Hyland2006,Levy2001,Plotkin2009,Pretnar2010}, 
which we have also applied in this paper.
%
In our approach, we have seen that the \Idris{} type system can express
effects by embedding, without any need to extend the language or type system. 
We make no attempt to \remph{infer} effect types, however, since our intention
is to \remph{specify} effects and \remph{check} that they are used correctly.
Nevertheless, since \texttt{EffM} is represented as an algebraic data type,
some effect inference would be possible.

%Also~\cite{Kammar2012}.

While our original motivation was to avoid the need for monad
transformers~\cite{Liang1995} in order to compose effects, there is a clear
relationship. Indeed, \texttt{EffM}
is in a sense a monad transformer itself, in that it may augment an underlying
monad with additional effects. Therefore, we can expect it to be possible
to combine effects and monad transformers, where necessary. 
The problem with modularity of monad transformers is well-known, and addressed
to some extent~\cite{Jaskelioff2009}, though this still does not allow easy
reordering of transformers, or reducing the transformer stack.
The \Eff{} approach encourages a more fine-grained separation of effects,
by making it easy to call functions which use a smaller set of effects.

Our approach, associating resources with each effect,
leads to a natural way of expressing and verifying resource usage protocols,
by updating the resource type. This is a problem previously tackled by
others, using special purpose type systems~\cite{Walker2000} 
or Haskell extensions~\cite{McBride2011}, and in my own previous
work~\cite{Brady2010a,bradyresource} by creating a DSL for resource management,
but these are less flexible than the present approach in that
combining resources is difficult. 
A related concept is
Typestate~\cite{Aldrich2009,Strom1986}, which similarly allows states to
be tracked in types, though again, we are able to implement this directly
rather than by extending the language. 

To some extent, we can now support imperative programming with dependent
types, such as supported by Xanadu~\cite{Xi2000} and Ynot~\cite{Ynot2008}. 
Ynot in particular is an axiomatic extension to Coq which allows reasoning
about imperative programs using Hoare Type Theory~\cite{HoareTT2008} ---
preconditions and postconditions on operations can be expressed in \Eff{} by
giving appropriate input and output resource types.
%
The difficulty in imperative programming with dependent types is that
updating one value may affect the type of another, though in our interpreter
example in Section \ref{sect:impint} we have safely used a dependent type
in a mutable state.

%(Point about typestate and effects: there are languages built around these
%concepts.  We have done neither, but built a language flexible enough to
%capture the same concepts.)

\section{Conclusions}

\label{sect:conclusion}

\Idris{} is a new language, with a full dependent type system. 
This gives us an ideal opportunity to revisit old problems
about how to handle effects in a pure functional language --- while the old
approach, based on monads and transformers, has proved successful in
Haskell, in this paper we have investigated an alternative approach 
found it to be a natural approach to defining and using effects.
By linking each effect with a \remph{resource} we can even track the state
of resources through program execution, and reason about resource usage
protocols such as file management. 

The \Eff{} approach has a number of strengths, and some weaknesses which
we hope to address in future work. The main strength is that many
common effects such as state, I/O and exceptions can be combined without
the need for monad transformers. Effects can be implemented independently,
and combined without furhter effort. Lifting is automatic --- sub-programs
using a smaller set of effects can be called without any explicit lifting
operation, so as well as being easy to combine effects, programs
remain readable. We have described a number of effects which are representable
in this setting, and there are several others we have not described which
are easy to define, such as parsing, logging, and bounded mutable arrays.
Arrays, statically bounded with a dependent type, could even lead to optimisations
since the effect system can guarantee that only one copy is needed, therefore
the array could have a compact representation with no run-time bounds
checking.

Another advantage is that it will be possible to have fine-grained separation
of systems effects --- we can be precise about needing network support, CGI,
graphics, the operating system's environment, etc, rather than including all of
these effects in one \texttt{IO} monad.

While using \Eff{}, we are still able to use monads.  Effects work \remph{with}
monads rather than against them, and indeed effectful programs are generally
translated to monadic
programs. As a result, concepts which need to be monadic (for example,
continuations) can remain so, and still work with \Eff{}.

\subsection*{Further Work}

The \Eff{} system is promising as a general approach to effectful programming,
but we have only just begun. At present, we are developing more libraries
using \Eff{} to assess how well it works in practice as a progamming tool,
as well as how efficient it is for realistic programs.

The most obvious weakness of \Eff{}, which is already known for the algebraic
approach, is that algebraic effects cannot capture all monads. This does not
appear to be a serious problem, however, given that \Eff{} is
designed to interact with monads, rather than to replace them. More seriously,
but less obviously, there is a small interpreter overhead since \texttt{EffM}
is represented as an algebraic data type, with an associated interpreter.  We
have not yet investigated in depth how to deal with this overhead, or indeed if
it is a serious problem in practice, but we expect that partial
evaluation~\cite{Brady2010} or a finally tagless approach~\cite{Carette2009}
would be sufficient. 

Another weakness which we hope to address is that mixing
control and resource effects is a challenge. For example, we cannot
currently thread state through all branches of a non-deterministic search.
If we can address this, it may be possible to represent more sophisticated
effects such as co-operative multithreading, or even partiality.
One possible way to tackle this problem would be to introduce a new method
of the \texttt{Handler} class, with a default implementation calling the
usual \texttt{handle}, which manages resources more precisely.

An implementation detail which could be improved without affect usage of
the library is that effectful sub-programs require ordering of effects
to be preserved. This should be a small improvement, merely requiring permuation
proofs of lists, but ideally generating these proofs will be automatic.

The \Eff{} implementation is entirely within the \Idris{} language --- no
extensions were needed. It takes advantage of dependent types and theorem
proving in several small but important ways: heterogenous lists of resources,
proofs of list membership and sub-list predicates, and parameterisation of
resources. Since modern Haskell supports many of these features, this leads
to an obvious question: what would it take to implement \Eff{} as an embedded
library in Haskell? An interesting topic for further study would be whether
this approach to combining effects would be feasible in a more mainstream
functional language.

Monad transformers have served us well, and are a good fit for the Haskell
type system. However, type systems are moving on. Perhaps now is the time to
look for something more effective.

