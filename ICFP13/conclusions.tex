\section{Related Work}

\label{sect:related}

DDC, Frank, Koka, Eff~\cite{Bauer}.

Unlike these other implementations, rather than building a language around
an algebraic effect system, we have taken the theory of algebraic effects and
implemented it directly in a general purpose programming language.

Relationship with monad transformers.  Earlier PADL work~\cite{bradyresource}.

Xi, imperative programming with dependent types~\cite{Xi2000}.
Typestate?~\cite{Strom1986,Aldrich2009}

(Point about typestate and effects: there are languages built around these
concepts.  We have done neither, but built a language flexible enough to
capture the same concepts.)

Theoretical work on effects: Plotkin, Power, Kammar, etc. Free monads?

\section{Conclusions}

\label{sect:conclusion}

Strengths: monad transformers not needed. Lift automatic, can permute effects
without need to change program. Flexibility. Easy to read and write.
Other effects we should be able to support --- parsing, logging, bounded
mutable arrays, and in particular
it will be possible to have fine-grained separation of systems effects, e.g.
we can be precise about needing network support, CGI, graphics, OS
environment, etc, rather than throwing it all into one IO monad.

We can still use the monadic implementations we know and love --- effects work
\remph{with} monads rather than against them, and indeed effectful programs
are (or at least can be) translated to monadic programs in the end. So things
which need to be monads still can be --- important since not every monad
can be translated to an effect.

Weaknesses: doesn't capture all monads (is this a weakness or just an
observation?). Small interpreter overhead, but could be erased either with
partial evaluation~\cite{Brady2010} or, possibly better, a finally
tagless approach~\cite{Carette2009}. Also need to ensure that the elaborator
can handle large programs --- it has to do a lot of work.

Mixing control and resource effects is still a bit awkward --- can't yet
thread state through all branches of a non-deterministic search, for example.
If we could do this, could possibly support co-operative multithreading with
shared state.

\subsection{Further Work}

See if it works in practice, then efficiency --- partial evaluation as in
~\cite{Brady2010} is not quite enough, but using case operators rather
than pattern matching ought to make it enough.

Another weakness to address --- representation of sublists requires ordering
to be preserved. It's easy to fix, since permuting a list is easy, but it
would be nice to be automatic.

Possibly making handle more flexible so that it can allow resources to interact
--- say, supporting shared state in co-operative multithreading.

What about doing this in Haskell? Recent type system extensions, plus type
classes for automatic proof construction. Do we really need monad transformers
for combining effects any more?

