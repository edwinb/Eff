\section{Introduction}

%Pure functional programming languages with \remph{dependent types} such as
%\Idris{}~\cite{idristutorial}, Agda~\cite{norell2009dependently} and
%Coq~\cite{Bertot2004} offer a programmer can both to write programs and verify
%specifications within the same framework. Thanks to the Curry-Howard
%correspondence, a program which type checks can be viewed as a proof that it
%conforms to the specification given by its type. Dependent types allow
%more precise types, and therefore more precise specifications.

Pure functions do not have side effects, but real applications do:
they may have state, communicate across a network, read and write files, 
or interact with users, among many other things. Furthermore, real systems
may fail --- data may be corrupted or untrusted. Pure functional programming
languages such as Haskell~\cite{Haskell98} manage such problems using
\remph{monads}~\cite{Wadler1995}, and allow multiple effects to be combined
using a stack of \remph{monad transformers}~\cite{Liang1995}.

Monad transformers are an effective tool for structuring larger Haskell
applications. A simple application using a Read-Execute-Print loop,
for example, may have some global state and perform console I/O, and hence
be built with an \texttt{IO} monad transformed into a state monad
using the \texttt{StateT} transformer. However, there are some difficulties
with building applications in this way. Two of the most important
are that the order in which transformers are applied matters (that is, 
transformers do not commute easily), and that it is difficult to invoke a
function which uses a subset of the transformers in the stack.
To illustrate these problem, consider an evaluator, in Haskell, for a 
simple expression language:

\begin{SaveVerbatim}{exprlang1}

data Expr = Val Int | Add Expr Expr

eval : Expr -> Int
eval (Val i) = i
eval (Add x y) = eval x + eval y

\end{SaveVerbatim}
\useverb{exprlang1}

\noindent
If we extend this language with variables, we need to extend the evaluator
with an environment, and deal with possible failure if a variable is undefined
(we omit the \texttt{Val} and \texttt{Add} cases):

\begin{SaveVerbatim}{exprlang2}

data Expr = Val Int | Add Expr Expr | Var String

type Env = [(String, Int)]

eval :: Expr -> ReaderT Env Maybe Int
eval (Var x) = do env <- ask
                  val <- lift (lookup x env)
                  return val

\end{SaveVerbatim}
\useverb{exprlang2}

\noindent
Here, the \texttt{Maybe} monad captures possible failure, and is transformed
into a reader monad using the \texttt{ReaderT} transformer to store the
environment, which is retrieved using \texttt{ask}. The \texttt{lift} operation
allows functions in the inner \texttt{Maybe} monad to be called.  
We can extend the language further, with random number generation:

\begin{SaveVerbatim}{exprlang3}

data Expr = Val Int | Add Expr Expr | Var String
          | Random Int

eval :: Expr -> RandT (ReaderT Env Maybe) Int
eval (Var x) = do env <- lift ask
                  val <- lift (lift (lookup x env))
                  return val
eval (Random x) = do val <- getRandomR (0, x)
                     return val

\end{SaveVerbatim}
\useverb{exprlang3}

\noindent
We have added another transformer to the stack, \texttt{RandT}, and added
\texttt{lift} where necessary to access the appropriate monads in the stack.
We have been able to build this interpreter from reusable components --- the
\texttt{Maybe} monad and \texttt{ReaderT} and \texttt{RandT} transformers ---
which is clearly a good thing. One small problem, however, is that the use of
\texttt{lift} is a little noisy, and will only get worse if we add more
monads to the stack, such as \texttt{IO}. Bigger problems occur if we need
to permute the order of the transformers, or invoke a function which uses a
subset, for example:

\begin{SaveVerbatim}{mtransprobs}

permute :: ReaderT Env (RandT Maybe) a -> 
           RandT (ReaderT Env Maybe) a
dropReader :: RandT Maybe a -> 
              RandT (ReaderT Env Maybe) a

\end{SaveVerbatim}
\useverb{mtransprobs}

\noindent
These problems mean that, in general, there is little motivation for separating
effects, and a temptation to build an application around one general purpose
monad capturing all of an application's state and exception handling needs.

It would be desirable, on the other hand, to separate effects into specific
components such as console I/O, file and network handling and operating system
interaction, for the same reason that it is desirable to separate the pure
fragments of a program from the impure fragments using the \texttt{IO} monad.
That is, the program's type would give more precise information about what the
program is supposed to do, making the program easier to reason about and
reducing the possibility of errors.

In this paper, I present an alternative approach to combining effects in
a pure functional programming language, based on handlers of
\remph{algebraic effects}~\cite{Bauer}, and
implemented directly as a domain specific language embedded
in a dependently typed host language, \Idris{}~\cite{Brady2013,idristutorial}.

\subsection{Contributions}

This paper takes as its hypothesis that algebraic effects provide a cleaner,
more composable and more flexible notation for programming with side effects
than monad transformers. Although they are not equivalent in power --- monads
and monad transformers can express more concepts --- many common effectful
computations are captured. 
The main contribution of this paper is a notation for describing and combining
side effects using \Idris{}. More specifically:

%, the paper makes the following contributions:

\begin{itemize}
\item An Embedded Domain Specific Language (DSL), \Eff{}, for capturing \emph{algebraic
effects}, in such a way that they are easily composable, and translatable to a
variety of underlying contexts using \emph{effect handlers}.
% \item An extension of the DSL for tracking the \emph{state} of resources
% associated with effects, such as whether a file handle is open or a network
% socket is ready to receive data.
\item A collection of example effects (State, Exceptions,
File and Console I/O, random number generation and
non-determinism) and their handlers. I show how alternative handlers
can be used to evaluate effects in different contexts. In particular, we can
implement unit tests for (and possibly reason about) IO programs by handling
the effect in the context of input and output streams.
\item I give example programs which combine effects, including a larger
example, an interpreter for an imperative language with mutable variables, to
illustrate how effectful applications may be structured.
\end{itemize}

\noindent
The \Eff{} DSL makes essential use of \emph{dependent types}, firstly to verify
that a specific effect is available to an effectful program using simple
automated theorem proving, and secondly to track the state of a resource by
updating its type during program execution. In this way, we can use the \Eff{}
DSL to verify implementations of resource usage protocols.

I describe how to \remph{use} \Eff{} in Section \ref{sect:effdsl},
how it is implemented in Section \ref{sect:effimpl}, and give a larger example
in Section \ref{sect:impint}. It is distributed with
\Idris{}\footnote{\url{http://idris-lang.org/}}. All of the examples in this
paper are available online at \url{http://idris-lang.org/effects}.



%[Idea: We can use the ability to implement different handlers to make
%unit tests for interactive programs. Worth doing? Also, possibly, reasoning
%about effectful programs.]

%\subsection{Outline}

% TODO: Drop this into the narrative above...

%This paper is structured as follows: In Section \ref{sect:idris} I briefly
%introduce the \Idris{} programming language, in particular the distinctive
%features we will use in the \Eff{} implementation; Section \ref{sect:effectsfp}
%motivates the problem of effect handling by describing an effectful evaluator
%implemented in \Idris{}; in Section \ref{sect:effdsl} I describe the \Eff{} DSL,
%showing how to \remph{use} defined effects, and how to define new effects;
%Section \ref{sect:effimpl} covers the implementation of \Eff{}; in Section
%\ref{sect:impint} I give a larger example, an interpreter for an imperative
%language implemented using \Eff{}, then finally I discuss related work and
%conclude in Sections \ref{sect:related} and \ref{sect:conclusion}.


%Possible structure:

%\begin{itemize}
%\item Monad transformers and why they don't compose nicely
%\item Motivating example in \Idris{}
%\item How to use it, how to create effects
%\item How it works
%\item Bigger example (CGI?)
%\end{itemize}

