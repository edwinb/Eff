\section{The \Eff{} DSL implementation}

\label{sect:effimpl}

\begin{SaveVerbatim}{effdslstruct}
data EffM : (m : Type -> Type) -> List EFFECT -> List EFFECT -> Type -> Type where
     return    : a -> EffM m es es a
     (>>=)     : EffM m es es' a -> (a -> EffM m es' es'' b) -> EffM m es es'' b
     mkEffectP : (prf : EffElem e a es) -> (eff : e a b t) -> EffM m es (updateResTy es prf eff) t
     liftP     : (prf : SubList fs es) -> EffM m fs fs' t -> EffM m es (updateWith fs' es prf) t
     (:-)      : (l : ty) -> EffM m [x] [y] t -> EffM m [l ::: x] [l ::: y] t
     new       : Handler e m => res -> EffM m (MkEff res e :: es) (MkEff res' e :: es') a -> EffM m es es' a
     catch     : Catchable m err => EffM m es es' a -> (err -> EffM m es es' a) -> EffM m es es' a
\end{SaveVerbatim}

\begin{figure*}[t]
\begin{center}
\useverb{effdslstruct}
\end{center}
\caption{The \Eff{} DSL data type}
\label{effdsltype}
\end{figure*}

The implementation of \Eff{} follows a common pattern in dependently typed DSL
implementation, that of the well-typed
interpreter~\cite{augustsson1999exercise,Brady2006a,Pasalic2002}. We describe
the DSL as a data type, \texttt{EffM}, which expresses the properties a program
must satisfy in its type, with a corresponding interpreter. The type system
guarantees that only programs which satisfy these properties 
can be interpreted.

\subsection{Language representation}

%The \Eff{} language is implemented as a syntax tree \texttt{EffM}
%plus a tagless interpreter, where the
%types in the syntax tree ensure that all \Eff{} programs are well typed.

The \texttt{EffM} data type represents the \Eff{} language constructs.
It is parameterised over its computation context \texttt{m},
and indexed by the list of effects on input and the list of effects on output,
as well as the return type of the computation:

\begin{SaveVerbatim}{effty}

data EffM : (m   : Type -> Type) -> 
            (es  : List EFFECT) -> 
            (es' : List EFFECT) -> (a : Type) -> Type

\end{SaveVerbatim}
\useverb{effty}

\noindent
For the common case of programs in which the input effects are the same as
the output effects, we define \texttt{Eff}:

\begin{SaveVerbatim}{effimmutable}

Eff : (m : Type -> Type) -> 
      (es : List EFFECT) -> (a : Type) -> Type
Eff m es t = EffM m es es t

\end{SaveVerbatim}
\useverb{effimmutable}

\noindent
The complete syntax is given in Figure \ref{effdsltype} for reference. In this
section, we describe the constructs in detail.

\textbf{Remark: } One way of looking at \texttt{EffM} is as a flexible monad
transformer, augmenting an underlying monad with additional features. In this way
we can combine algebraic effects with concepts more difficult to capture, such
as partiality and continuations.

\subsubsection{Basic constructs}

In the simplest case, we would like to inject pure values into the
\texttt{EffM} representation:

\begin{SaveVerbatim}{effval}

return : a -> EffM m es es a

\end{SaveVerbatim}
\useverb{effval}

\noindent
Additionally, we have \texttt{(>>=)} to support \texttt{do} notation:

\begin{SaveVerbatim}{effbind}

(>>=) : EffM m es es' a -> 
        (a -> EffM m es' es'' b) -> EffM m es es'' b

\end{SaveVerbatim}
\useverb{effbind}

\noindent
The type of \texttt{(>>=)} expresses that available effects may change,
transitively, across a program. The output of the first action, \texttt{es'}
becomes the input to the second.

\textbf{Remark:} \Idris{} allows ad-hoc name overloading, as well as overloading via type classes.
Since it does not support type inference at the top level, as full type inference
is undecidable for dependently typed languages, such overloading can be
resolved in a type directed way.
It is important here, since we cannot make
\texttt{EffM} an instance of \texttt{Monad}, but we would like to use
\texttt{do}-notation as a natural way of implementing imperative code.

\subsubsection{Invoking effects}

Recall that an algebraic effect is promoted to an \Eff{} program 
with the \texttt{mkEffect} function, the type of which has so far
been postponed. We can invoke effects using a helper, \texttt{mkEffectP}, which
requires a \remph{proof}, \texttt{prf},
that the effect 
%we are invoking 
is available:

\begin{SaveVerbatim}{effinvoke}

mkEffectP : (prf : EffElem e a es) -> 
            (eff : e a b t) -> 
            EffM m es (updateResTy es prf eff) t

\end{SaveVerbatim}
\useverb{effinvoke}

\noindent
If we are invoking an effect \texttt{eff}, provided by the algebraic effect
\texttt{e}, with an input resource type \texttt{a},
this is valid only if we can provide a proof that the algebraic effect is
present in the input set of resources \texttt{es}. This concept is captured
in the \texttt{EffElem} predicate:

\begin{SaveVerbatim}{effelem}

data EffElem : (Type -> Type -> Type -> Type) -> 
               Type -> List EFFECT -> Type where
     Here : EffElem x a (MkEff a x :: es)
     There : EffElem x a es -> EffElem x a (y :: es)

\end{SaveVerbatim}
\useverb{effelem}

\noindent
This proof serves two purposes --- it provides a guarantee that the effect
is available with the correct resource type, and, as we shall see shortly,
it serves as an index into the resource set when we \remph{evaluate} programs.
%and compute the list of output resources.
%
After the effect has been handled, its resource type may have been updated.
We must therefore update the output resource state in the type. Again, we use
the structure of the proof, to update the resource type from \texttt{a} to
\texttt{b}.

\begin{SaveVerbatim}{updateresty}

updateResTy : (es : List EFFECT) -> 
              EffElem e a es -> e a b t -> List EFFECT
updateResTy {b} (MkEff a e :: es) Here n 
    = (MkEff b e) :: es
updateResTy (x :: es) (There p) n 
    = x :: updateResTy es p n

\end{SaveVerbatim}
\useverb{updateresty}

\noindent
The problem with \texttt{mkEffectP} is that we must provide an 
explicit proof that the effect is available. Since the effect list is statically
known, \Idris{} ought to be able to find such a proof automatically.

\Idris{} currently has limited proof search, based on
reflection and reification of programs, but it
is sufficient for constructing the required proofs. The reflection mechanism
itself is beyond the scope of this paper. Briefly, we have
the following function, which constructs a tactic for searching for proofs 
of \texttt{EffElem}:

\begin{SaveVerbatim}{findeff}

findEffElem : Tactic 

\end{SaveVerbatim}
\useverb{findeff}

\noindent
Then, we have a notation for giving \remph{default} implicit arguments to
functions. The \texttt{mkEffect} function is a wrapper for \texttt{mkEffectP}
with a default argument which invokes the \texttt{findEffElem} tactic. If
the proof search fails, this causes a compile time error, reporting that the
effect is not available.

\label{sect:mkeffect}
\begin{SaveVerbatim}{impeff}

mkEffect : {default tactics { reflect findEffElem; } 
              prf : EffElem e a es} -> 
           (eff : e a b t) -> 
           EffM m es (updateResTy es prf eff) t
mkEffect {prf} e = mkEffectP prf e

\end{SaveVerbatim}
\useverb{impeff}

\noindent
Fortunately, there is no need for a user of the library to know anything about
this proof search mechanism. Tactic construction for automated proof search is
related to the \texttt{Ltac}~\cite{Delahaye2000} system in Coq, the intention
being to make simple proofs automatable.

\subsubsection{Effectful sub-programs}

As well as invoking algebraic effects directly, we would like to be able to
invoke sub-programs, which may use all of the effects available, or
a subset. To achieve this, we use the \texttt{liftP} construct:

\begin{SaveVerbatim}{efflift}

liftP : (prf : SubList fs es) ->
        (prog : EffM m fs fs' t) -> 
        EffM m es (updateWith fs' es prf) t

\end{SaveVerbatim}
\useverb{efflift}

\noindent
This requires a proof that the effects available to the sub-program \texttt{prog}
are a subset (strictly, a sub-list) of the effects available to the outer program,
expressed using the following predicate:

\begin{SaveVerbatim}{sublist}

data SubList : List a -> List a -> Type where
     SubNil : SubList [] []
     Keep   : SubList es fs -> 
              SubList (x :: es) (x :: fs)
     Drop   : SubList es fs -> 
              SubList es (x :: fs)

\end{SaveVerbatim}
\useverb{sublist}

\noindent
This predicate describes a sublist in terms of a larger list by saying, for each
element of the larger list, whether it is kept or dropped. An alternative
base case could state that \texttt{[]} is a sub-list of any list, assuming
that remaining items are dropped.

The sub-program may itself update the effects it uses, so again the proof serves
two purposes: Firstly, to ensure the effects are indeed available; and secondly,
to be able to update the effects in the outer program once the sub-program is
complete, as follows:

\begin{SaveVerbatim}{updwith}

updateWith : (fs' : List a) -> (es : List a) ->
             SubList fs es -> List a
updateWith (y :: fs) (x :: es) (Keep rest) 
           = y :: updateWith fs es rest
updateWith fs        (x :: es) (Drop rest) 
           = x :: updateWith fs es rest
updateWith []        []        SubNil      = []

\end{SaveVerbatim}
\useverb{updwith}

\noindent
Again, we can construct the necessary proofs of \texttt{SubList} automatically,
if the sub-program uses a valid set of effects, because all effects are
statically known, using a reflected tactic \texttt{findSubList}. We implement
a wrapper \texttt{lift} which builds this proof implicitly:

\begin{SaveVerbatim}{liftimp}

implicit
lift : {default tactics { reflect findSubList; }
          prf : SubList fs es} ->
       (prog : EffM m fs fs' t) -> 
       EffM m es (updateWith fs' fs prf) t
lift {prf} e = lift prf e

\end{SaveVerbatim}
\useverb{liftimp}

\noindent
The \texttt{implicit} modifier before \texttt{lift} states that this function
can be used as an \remph{implicit conversion}. Implicit conversions are inspired
by a similar feature in Scala~\cite{Scala} --- the effect of the \texttt{implicit}
modifier is, intuitively, that \texttt{lift} will be applied to
a program in \texttt{EffM} if it is required for type correctness. 
%It is
%simple to check if an implicit conversion is required during type 
%checking --- 
Since type checking is type directed it always has access to the
required type of an expression, and the implicit coercions which produce
that type, so applying conversions is simple.

Such conversions are, deliberately, limited. They cannot be chained, unlike
implicit coercions in Coq, to avoid coherence problems. Furthermore, to avoid
ambiguity problems, if there is more than one implicit conversion available
then \remph{neither} will be applied. In the \Eff{} library, only \texttt{lift}
is implicit.

\textbf{Remark:} Using \texttt{SubList} as it is, rather than some notion
of subset, means that in sub-programs the effects must appear in the same order
as they appear in the caller's effect list. This is not an inherent limitation,
--- with improved proof search, we should also be able to support
effect sets which are \remph{permutations} of another.

\subsubsection{Labelling effects}

If we have an effectful program \texttt{p} with a single effect, we can 
\remph{label} that effect using the \texttt{(:-)} operator:

\begin{SaveVerbatim}{lblintro}

(:-)  : (l : ty) -> 
        EffM m [x] [y] t -> 
        EffM m [l ::: x] [l ::: y] t

\end{SaveVerbatim}
\useverb{lblintro}

\noindent
It is sufficient to handle single effects here, because labels can only
apply to one effect at a time, and such effectful sub-programs can easily
be invoked using an implicit \texttt{lift} as described above. Labelling
effects does nothing more than adding a label to the effect and its corresponding
resource.

\subsubsection{Introducing effects}

We can introduce a new effect in the course of an effectful program, provided
that the effect can be handled in the current computation context \texttt{m}:

\begin{SaveVerbatim}{neweff}

new : Handler e m => 
      res -> EffM m (MkEff res e :: es) 
                    (MkEff res' e :: es') a ->
      EffM m es es' a

\end{SaveVerbatim}
\useverb{neweff}

\noindent
This extends the list of effects, initialised with
a resource of type \texttt{res}.
Once the sub-program is complete, the resource for the new effect is discarded,
as is clear from the type of \texttt{new}.
%
The effect itself, \texttt{e}, is never given explicitly here, which
means that it must be clear from the sub-program what the effect is. Typically,
this means that the sub-program will be a function with an explicit type.

\subsubsection{Handling failure}

Finally, if the computation context \texttt{m} supports failure handling,
we can use the \texttt{catch} construct to handle errors:

\begin{SaveVerbatim}{catcheff}

catch : Catchable m err =>
        (prog : EffM m es es' a) -> 
        (handler : err -> EffM m es es' a) ->
        EffM m es es' a

\end{SaveVerbatim}
\useverb{catcheff}

\noindent
If the sub-program \texttt{prog} fails with an error, of type \texttt{err}, then
the handler is called, being passed the error. Note that both \texttt{prog}
and \texttt{handler} transform the effect list from \texttt{es} to \texttt{es'}.
If \texttt{prog} fails, then the resources are reset to the state there were in
at the start. This requires some care, if the effect refers to external
resources such as file handles.

\subsection{The \Eff{} interpreter}

\begin{SaveVerbatim}{effdslinterp}
effInt : Env m es -> EffM m es es' a -> (Env m es' -> a -> m b) -> m b
effInt env (return x) k = k env x
effInt env (prog >>= c) k = effInt env prog (\env', p' => effInt env' (c p') k)
effInt env (mkEffectP prf effP) k = execEff env prf effP k
effInt env (liftP prf effP) k = let env' = dropEnv env prf in 
                                    effInt env' effP (\envk, p' => k (rebuildEnv envk prf env) p')
effInt env (new r prog) k = let env' = r :: env in 
                                effInt env' prog (\ (v :: envk), p' => k envk p')
effInt env (catch prog handler) k = catch (effInt env prog k)
                                          (\e => effInt env (handler e) k)
effInt env (l :- prog) k = let env' = unlabel env in
                                      effInt env' prog (\envk, p' => k (relabel l envk) p')
\end{SaveVerbatim}

\begin{figure*}[t]
\begin{center}
\useverb{effdslinterp}
\end{center}
\caption{The \Eff{} DSL interpreter}
\label{effdslimp}
\end{figure*}

Running an effectful program, of type \texttt{EffM m es es' t}, should yield
a computation of type \texttt{m t}, that is, the program returns a value of type
\texttt{t} in some computation context \texttt{m}. We can interpret programs
mostly through a simple traversal of the \texttt{EffM} syntax, subject to the
following considerations:

\begin{itemize}
\item We need to keep track of \remph{resources} corresponding to each effect.
\item We need to invoke the appropriate handlers where necessary.
\item We need to return two values as part of the computation: the result
\texttt{t} and an updated collection resources.
\end{itemize}

\noindent
To keep track of resources, we build an environment as a heterogenous list to
store the resource corresponding to each effect:

\begin{SaveVerbatim}{resenv}

data Env  : (m : Type -> Type) -> 
            List EFFECT -> Type where
     Nil  : Env m Nil
     (::) : Handler eff m => 
            a -> Env m es -> Env m (MkEff a eff :: es)

\end{SaveVerbatim}
\label{sect:envdef}
\useverb{resenv}

\noindent
Using the (overloaded) \texttt{Nil} and \texttt{(::)} gives us access to list
syntax for environments. They are parameterised over a computation context
\texttt{m}, which allows an effect handler instance to be associated with each
entry in the context. This is important both because it means \texttt{EffM}
programs can be independent of context, thus interpretable in several
contexts, and because effects and hence their handlers may change during 
execution.

Since we need to return two values, a result and an updated resource collection,
we implement the evaluator in continuation passing style, with the two values
passed to the continuation. This also helps when invoking handlers, which also
require a continuation. The interpreter has the following type:

\begin{SaveVerbatim}{cpsinterp}

effInt : Env m es -> EffM m es es' a -> 
         (Env m es' -> a -> m b) -> m b

\end{SaveVerbatim}
\useverb{cpsinterp}

\noindent
This takes an input set of resources, and a program, and the continuation to run
on the result and updated environment. Effectful programs are invoked using
a function of the following form, calling \texttt{effInt}
with a continuation which simply discards the environment when evaluation is
complete.

\begin{SaveVerbatim}{runeff}

run : Applicative m => 
      Env m es -> EffM m es es' a -> m a
run env prog = effInt env prog (\env, r => pure r)

runPure : Env id es -> EffM id es es' a -> a
runPure env prog = effInt env prog (\env, r => r)

\end{SaveVerbatim}
\useverb{runeff}

\noindent
The full implementation of \texttt{effInt} is given for reference in
Figure \ref{effdslimp}. The interpreter uses a number of helpers in order
to manage effects. Firstly, to invoke a handler given a proof that an effect
is available, we use \texttt{execEff}, defined as follows:

\noindent
\begin{SaveVerbatim}{invokehandler}

 execEff : Env m es -> (p : EffElem e res es) -> 
           (eff : e res b a) ->
           (Env m (updateResTy es p eff) -> a -> m t) -> 
           m t
 execEff (val :: env) Here eff' k 
     = handle val eff' (\res, v => k (res :: env) v)
 execEff (val :: env) (There p) eff k 
     = execEff env p eff (\env', v => k (val :: env') v)

\end{SaveVerbatim}
\useverb{invokehandler}

\noindent
The proof is used to locate the effect and handler in the environment.
Following the structure of the interpreter and the handlers,
\texttt{execEff} is written in continuation passing style. We can use this
directly to execute effects:

\begin{SaveVerbatim}{effeff}

effInt env (mkEffectP prf effP) k 
                   = execEff env prf effP k

\end{SaveVerbatim}
\useverb{effeff}

\noindent
Invoking a sub-program with a smaller collection of effects involves dropping
the unnecessary resources from the environment, invoking the sub-program, then
rebuilding the environment reinstating the resources which were not used.
We drop resources from an environment according to the 
\texttt{SubList} proof with \texttt{dropEnv}:

\begin{SaveVerbatim}{dropenv}

dropEnv : Env m fs -> SubList es fs -> Env m es
dropEnv [] SubNil = []
dropEnv (v :: vs) (Keep rest) = v :: dropEnv vs rest
dropEnv (v :: vs) (Drop rest) = dropEnv vs rest

\end{SaveVerbatim}
\useverb{dropenv}

\noindent
Correspondingly, \texttt{rebuildEnv} rebuilds the outer environment, updating
the resources which were updated by the sub-programs:

\begin{SaveVerbatim}{rebuildenv}

rebuildEnv : Env m fs' -> (prf : SubList fs es) -> 
             Env m es -> Env m (updateWith fs' es prf) 
rebuildEnv []        SubNil      env = env
rebuildEnv (f :: fs) (Keep rest) (e :: env) 
      = f :: rebuildEnv fs rest env
rebuildEnv fs        (Drop rest) (e :: env) 
      = e :: rebuildEnv es rest env

\end{SaveVerbatim}
\useverb{rebuildenv}

\noindent
We can use \texttt{dropEnv} and \texttt{rebuildEnv} to interpret the execution
of sub-programs, dropping resources before invoking, then rebuilding the
environment before invoking the continuation:

\begin{SaveVerbatim}{efflift}

effInt env (liftP prf effP) k 
  = let env' = dropEnv env prf in 
        effInt env' effP (\envk, p' => 
              k (rebuildEnv envk prf env) p')

\end{SaveVerbatim}
\useverb{efflift}

\noindent
To introduce a new effect using \texttt{new}, we simply extend the environment
with the new resources before invoking the sub-program, and drop the extra
resource before invoking the continuation:

\begin{SaveVerbatim}{effnew}

effInt env (new r prog) k 
     = let env' = r :: env in 
           effInt env' prog 
               (\ (v :: envk), p' => k envk p')

\end{SaveVerbatim}
\useverb{effnew}

\noindent
Interpreting a \texttt{catch} involves using the \texttt{catch} method of
the \texttt{Catchable} class. We rely on type directed overloading here to
disambiguate the \texttt{catch} in the \texttt{EffM} structure from the
\texttt{catch} provided by the \texttt{Catchable} class:

\begin{SaveVerbatim}{effcatch}

effInt env (catch prog handler) k 
     = catch (effInt env prog k)
             (\e => effInt env (handler e) k)

\end{SaveVerbatim}
\useverb{effcatch}

\noindent
Finally, to interpret a labelled effect, we remove the label, interpret the
resulting effect as normal, then replace the label (\texttt{unlabel} and
\texttt{label} have the obvious definitions, with \texttt{unlabel} removing
a label from a singleton environment, and \texttt{relabel} replacing it):

\begin{SaveVerbatim}{efflbl}

effInt env (l :- prog) k 
   = let env' = unlabel env in
         effInt env' prog (\envk, p' => 
                              k (relabel l envk) p')
\end{SaveVerbatim}
\useverb{efflbl}

