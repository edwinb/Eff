\section{\Eff{}: an Embedded DSL for Effects Management} 

In this section, I introduce \Eff{}, an embedded domain specific language
for managing computational effects in \Idris{}. I will introduce specific
distinctive features of \Idris{} as required --- in particular, we will
use implicit conversions and default implicit arguments in the implementation
of \Eff{} --- a full tutorial is available elsewhere~\cite{idristutorial}.  
First, I describe how to
use effects which are already defined in the language in order to
implement the evaluator described in the introduction. Then, I show how new
effects may be implemented. 

The framework consists of a DSL representation \texttt{Eff} for combining
effects, \texttt{EffM} for combining mutable effects, and implementations
of several predefined effects. We refer to the whole framework with the
name \Eff{}.

\label{sect:effdsl}

\subsection{Programming with \Eff{}}

Programs in the \Eff{} language are described using the following data type,
in the simplest case:

\begin{SaveVerbatim}{efftype}

Eff : (m  : Type -> Type) -> 
      (es : List EFFECT) -> (a  : Type) -> Type

\end{SaveVerbatim}
\useverb{efftype}

\noindent
Note that function types in \Idris{} take the form \texttt{(x : a) -> b}, with
an optional name \texttt{x} on the domain of the function. This is primarily
to allow the name \texttt{x} to be used in the codomain, although it is also
useful for documenting the purpose of an argument.

\texttt{Eff} is parameterised over a \remph{computation context}, \texttt{m}, which
describes the context in which the effectful program will be run, a list
of side effects \texttt{es} that the program is permitted to use, 
and the program's return type \texttt{a}. The name \texttt{m} for the computation
context is suggestive of a monad,
%--- the computation context most commonly is a monad 
but there is no requirement for it to be so.

Side effects are described using the \texttt{EFFECT} type --- we will refer
to these as \remph{concrete} effects. For example:

\begin{SaveVerbatim}{effs}

STATE     : Type -> EFFECT
EXCEPTION : Type -> EFFECT
FILEIO    : Type -> EFFECT
STDIO     : EFFECT
RND       : EFFECT

\end{SaveVerbatim}
\useverb{effs}

\noindent
States are parameterised by the type of the state being carried, and exceptions
are parameterised by a type representing errors. File I/O is an effect which
allows a single file to be processed, with the type giving the current state
of the file (i.e. closed, open for reading, or open for writing). The
\texttt{STDIO} effect permits console I/O, and \texttt{RND} permits random
number generation.
%
For example, a program with some integer state, which performs console I/O 
and which could throw
an exception of type \texttt{Error} might have the following type:

\noindent
\begin{SaveVerbatim}{exprog}

 example : Eff IO [EXCEPTION Error, STDIO, STATE Int] ()

\end{SaveVerbatim}
\useverb{exprog}

\noindent
More generally, a program might modify the set of effects available. This
might be desirable for several reasons, such as adding a new effect, or to
update an index of a dependently typed state. In this case, we describe
programs using the \texttt{EffM} data type:

\begin{SaveVerbatim}{effmtype}

EffM : (m   : Type -> Type) -> 
       (es  : List EFFECT) -> 
       (es' : List EFFECT) -> 
       (a   : Type) -> Type

\end{SaveVerbatim}
\useverb{effmtype}

\noindent
\texttt{EffM} is parameterised over the context and type as before,
but separates input effects (\texttt{es}) from output effects (\texttt{es'}).
In fact, \texttt{Eff} is defined in terms of \texttt{EffM}, with equal
input/output effects.

We adopt the convention that the names \texttt{es} and \texttt{fs} refer to a list
in input effects, and \texttt{es'} and \texttt{fs'} refer to a list
of output effects.

\subsubsection{First example: State}

In general, an effectful program implemented in the \texttt{EffM} structure has
the look and feel of a monadic program in Haskell, since \texttt{EffM} supports
\texttt{do}-notation. To illustrate basic usage, let us implement
a stateful function, which tags each node in a binary tree with
a unique integer, depth first, left to right. We declare trees as
follows:

\begin{SaveVerbatim}{treedef}

data Tree a = Leaf 
            | Node (Tree a) a (Tree a)

\end{SaveVerbatim}
\useverb{treedef}

\noindent
To tag each node in the tree, we write an effectful program which, for each
node, tags the left subtree, reads and updates the state, tags the right
subtree, then returns a new node with its value tagged. The type
states that the program requires an integer state:

\begin{SaveVerbatim}{treelblty}

tag : Tree a -> Eff m [STATE Int] (Tree (Int, a))

\end{SaveVerbatim}
\useverb{treelblty}

\noindent
The implementation traverses the tree, using \texttt{get} and \texttt{put}
to manipulate state:

\begin{SaveVerbatim}{treelbl}

tag Leaf = return Leaf
tag (Node l x r) 
     = do l' <- tag l
          lbl <- get; put (lbl + 1)
          r' <- tag r
          return (Node l' (lbl, x) r')

\end{SaveVerbatim}
\useverb{treelbl}

\noindent
The \Eff{} system ensures, statically, that any
effectful functions which are called (\texttt{get} and \texttt{put} here)
require no more effects than are available.
The types of these functions are:

\begin{SaveVerbatim}{getputty}

get : Eff m [STATE x] x
put : x -> Eff m [STATE x] ()

\end{SaveVerbatim}
\useverb{getputty}

\noindent
Each effect is associated with a \remph{resource}. For example, the resource
associated with \texttt{STATE Int} is the integer state itself.  To \remph{run}
an effectful program, we must initialise each resource and instantiate
\texttt{m}. Here we instantiate \texttt{m} with \texttt{id}, resulting in a
pure function.

\begin{SaveVerbatim}{runEval}

tagFrom : Int -> Tree a -> Tree (Int, a)
tagFrom x t = runPure [x] (tag t)

\end{SaveVerbatim}
\useverb{runEval}

\noindent
In general, to run an effectful program, we use one of the functions
\texttt{run}, \texttt{runWith} or \texttt{runPure}, instantiating an
environment with resources corresponding to each effect:

\begin{SaveVerbatim}{runeffs}

run     : Applicative m => 
          Env m es -> EffM m es es' a -> m a
runWith : (a -> m a) -> 
          Env m es -> EffM m es es' a -> m a
runPure : Env id es -> EffM id es es' a -> a

\end{SaveVerbatim}
\useverb{runeffs}

\noindent
Corresponding functions \texttt{runEnv}, \texttt{runWithEnv} and
\texttt{runPureEnv} are also available for cases when the final resources are
required.  The reason \texttt{run} needs \texttt{m} to be an applicative
functor
is that it uses \texttt{pure} to inject a pure value into \texttt{m}. If this
is inconvenient, \texttt{runWith} can be used instead. Unlike the
monad transformer approach, there is no requirement that
\texttt{m} is a monad. Any type transformer is fine --- in particular,
\texttt{id} can be used if
the effectful program can be translated into a pure function.
%
As we will see, the particular choice of \texttt{m} can be
important. Programs with exceptions, for example, can be run
in the context of \texttt{IO}, \texttt{Maybe} or \texttt{Either}.
%
We will return to the definition of \texttt{Env} in Section \ref{sect:envdef}.
% For the moment, 
For now, it suffices to know that it is a heterogeneous list of values
for the initial resources \texttt{es}.

\subsubsection{Labelled Effects}

When we invoke an effect, %such as \texttt{get} or \texttt{put},
\Eff{} internally searches the list of available effects to check
it is supported, and invokes the effect using the corresponding
resource. This leads to an important question: what if we have more than one 
effect which supports the function? A particular situation where this arises
is when we have more than one integer state.
%
For example, to count the number of \texttt{Leaf} nodes
in a tree while tagging nodes, we will need two integer states:

\begin{SaveVerbatim}{treelblcount}

tagCount : Tree a -> 
     Eff m [STATE Int, STATE Int] (Tree (Int, a))

\end{SaveVerbatim}
\useverb{treelblcount}

\noindent
What should be the effect of \texttt{get} and \texttt{put} in this program?
Which effect do they act on?
%Do they read and update the first state, the second, or choose
%non-deterministically? 
%
In practice, the earlier effect is chosen. While clearly defined, this is
unlikely to be the desired behaviour, so 
to avoid this problem, effects may also be \remph{labelled} using the
\texttt{:::} operator.  A label can be of any type, and an
effect can be converted into a labelled effect using the \texttt{:-}
operator:

\begin{SaveVerbatim}{lbleff}

(:::) : lbl -> EFFECT -> EFFECT
(:-)  : (l : lbl) -> EffM m [x] [y] t -> 
                     EffM m [l ::: x] [l ::: y] t

\end{SaveVerbatim}
\useverb{lbleff}

\noindent
In order to implement \texttt{tagCount} now, first we define a type for the
labels. We have one state variable representing the leaf count, and one
representing the current tag:

\begin{SaveVerbatim}{lbltys}

data Vars = Count | Tag

\end{SaveVerbatim}
\useverb{lbltys}

\noindent
Then, we use these labels to disambiguate the states. To increment the count
at each leaf, we use \texttt{update}, which combines a \texttt{get} and a
\texttt{put} by applying a function to the state:

\begin{SaveVerbatim}{treelblcountty}

tagCount : Tree a -> Eff m [Tag   ::: STATE Int, 
                            Count ::: STATE Int] 
                              (Tree (Int, a))
\end{SaveVerbatim}
\begin{SaveVerbatim}{treelblcountdef}
tagCount Leaf
     = do Count :- update (+1)
          return Leaf
tagCount (Node l x r) 
     = do l' <- tagCount l
          lbl <- Tag :- get
          Tag :- put (lbl + 1)
          r' <- tagCount r
          return (Node l' (lbl, x) r')

\end{SaveVerbatim}

\useverb{treelblcountty}

\useverb{treelblcountdef}

\noindent
In order to get the count, we will need access to the environment \remph{after}
running \texttt{tagCount}. To do so, we use \texttt{runPureEnv}, which returns
the final resource states as well as the result:

\begin{SaveVerbatim}{runenvty}

runPureEnv : Env id xs -> 
             EffM id xs xs' a -> (Env id xs', a)

\end{SaveVerbatim}
\useverb{runenvty}

\noindent
To initialise the environment, we give the label name along with the initial
value of the resource:

\begin{SaveVerbatim}{rpenvtree}

runPureEnv [Tag := 0, Count := 0] (tagCount t) 

\end{SaveVerbatim}
\useverb{rpenvtree}

\noindent
And finally, to implement a pure wrapper function which returns a pair of the
count of leaves and a labelled tree, we call \texttt{runPureEnv} with the
initial resources, and match on the returned resources to retrieve the leaf
count:

\begin{SaveVerbatim}{lcfrom}

tagCountFrom : Int -> Tree a -> (Int, Tree (Int, a))
tagCountFrom x t 
    = let ([_, Count := leaves], tree) =
       runPureEnv [Tag := 0, Count := 0] (tagCount t)
          in (leaves, tree)

\end{SaveVerbatim}
\useverb{lcfrom}

\subsubsection{An Effectful Evaluator revisited}

To implement the effectful evaluator from the introduction 
in \Eff{}, we support exceptions, random numbers and an environment
mapping from \texttt{String} to \texttt{Int}:

\begin{SaveVerbatim}{langenv}

Vars : Type
Vars = List (String, Int)

\end{SaveVerbatim}
\useverb{langenv}

\noindent
The evaluator invokes supported effects where needed. We use the following
effectful functions:

\begin{SaveVerbatim}{efftypes}

get    : Eff m [STATE x] x
raise  : a -> Eff m [EXCEPTION a] b
rndInt : Int -> Int -> Eff m [RND] Int

\end{SaveVerbatim}
\useverb{efftypes}

\noindent
The evaluator itself is written as an instance of \texttt{Eff}:
%type.
%, using
%\texttt{do}-notation as in the monad transformer implementation. 
%The 
%type, again, lists the resources we require:

\begin{SaveVerbatim}{langeffty}

eval : Expr -> 
       Eff m [EXCEPTION String, RND, STATE Vars] t

\end{SaveVerbatim}
\useverb{langeffty}

\noindent
The implementation simply invokes the required effects 
%\texttt{get}, \texttt{raise} and \texttt{rndInt}, 
with \Eff{} checking that these effects are available:

\begin{SaveVerbatim}{langeff}

eval (Val x) = return x
eval (Var x) = do vs <- get
                  case lookup x vs of
                    Nothing => raise ("Error " ++ x)
                    Just val => return val
eval (Add l r) = [| eval l + eval r |]
eval (Random upper) = rndInt 0 upper

\end{SaveVerbatim}
\useverb{langeff}

%\begin{figure}[h]
%\label{langeffint}
%\caption{An Effectful Interpreter}
%\end{figure}

\noindent
\textbf{Remark:}
We have used idiom brackets~\cite{McBride2007} in this implementation, to
give a more concise notation for applicative programming with effects.
An application inside idiom brackets, \texttt{[| f a b c d |]} translates
directly to:

\begin{SaveVerbatim}{idiomtrans}

pure f <$> a <$> b <$> c <$> d

\end{SaveVerbatim}
\useverb{idiomtrans}

\noindent
In order to run this evaluator, we must provide initial values for the resources
associated with each effect. Exceptions require the unit resource, random
number generation requires an initial seed, and the state requires an initial
environment. We instantiate \texttt{m} with \texttt{Maybe} to be able
to handle exceptions:

\begin{SaveVerbatim}{exprrun}

runEval : List (String, Int) -> Expr -> Maybe Int
runEval env expr = run [(), 123456, env] (eval expr)

\end{SaveVerbatim}
\useverb{exprrun}

\noindent
Extending the evaluator with a new effect, such as \texttt{STDIO} is a matter
of extending the list of available effects in its type.  We could use this, for
example, to print out the generated random numbers:

\begin{SaveVerbatim}{langeffio}

eval : Expr -> 
       Eff m [EXCEPTION String, STDIO, 
              RND, STATE Vars] t
...
eval (Random upper) = do num <- rndInt 0 upper
                         putStrLn (show num)
                         return num

\end{SaveVerbatim}
\useverb{langeffio}

\noindent
We can insert the \texttt{STDIO} effect anywhere in the list without difficulty
--- the only requirements are that its initial resource is in the corresponding
position in the call to \texttt{run}, and that \texttt{run} instantiates
a context which supports \texttt{STDIO}, such as \texttt{IO}:

\begin{SaveVerbatim}{exprrun}

runEval : List (String, Int) -> Expr -> IO Int
runEval env expr 
    = run [(), (), 123456, env] (eval expr)

\end{SaveVerbatim}
\useverb{exprrun}
\subsection{Implementing effects}

In order to implement a new effect, we define a new type (of kind \texttt{Effect})
and explain how that effect is interpreted in some underlying context
\texttt{m}. An \texttt{Effect} describes an effectful computation,
parameterised by an input resource \texttt{res}, an output resource \texttt{res'}, 
and the type of the computation \texttt{t}.

\begin{SaveVerbatim}{effectty}

Effect : Type
Effect = (res : Type) -> (res' : Type) -> 
         (t : Type) -> Type

\end{SaveVerbatim}
\useverb{effectty}

\noindent
Effects are typically described as algebraic data types. To \remph{run} an
effect, they must be handled in some specific computation context \texttt{m}.
We achieve this by making effects and contexts instances of a a type class,
\texttt{Handler}, which has a method \texttt{handle} explaining this
interpretation:

\begin{SaveVerbatim}{effh}

class Handler (e : Effect) (m : Type -> Type) where
     handle : res -> (eff : e res res' t) -> 
              (res' -> t -> m a) -> m a

\end{SaveVerbatim}
\useverb{effh}

\noindent
Type classes in \Idris{} may be parameterised by anything --- not only types,
but also values, and even other type classes. Thus, if a parameter is anything
other than a \texttt{Type}, it must be given a type label explicitly, like
\texttt{e} and \texttt{m} here.

Handlers are parameterised by the effect they handle, and the context in which
they handle the effect. This allows several different context-dependent
handlers to be written --- e.g. exceptions could be handled differently in an
\texttt{IO} setting than in a \texttt{Maybe} setting. When effects are combined,
as in the evaluator example, all effects must be handled
in the context in which the program is run.

An effect \texttt{e res res' t} updates a resource type \texttt{res} to a
resource type \texttt{res'}, returning a value \texttt{t}. The handler, therefore,
implements this update in a context \texttt{m} which may support side effects.
The handler is written in continuation passing style. This is for two reasons:
Firstly, it returns two values, a new resource and the result of the computation,
which is more cleanly managed in a continuation than by returning a tuple;
secondly, and more importantly, it gives the handler the flexibility to invoke
the continuation any number of times (zero or more).

An \texttt{Effect}, which is the internal algebraic description of an effect,
is promoted into a concrete \texttt{EFFECT}, which is expected by the
\texttt{EffM} structure, with the \texttt{MkEff} constructor:

\begin{SaveVerbatim}{efft}

data EFFECT : Type where
     MkEff : Type -> Effect -> EFFECT

\end{SaveVerbatim}
\useverb{efft}

\noindent
\texttt{MkEff} additionally records the resource state of an effect.
In the remainder of this section, we describe how several effects can be
implemented in this way: mutable state; console I/O; exceptions; files; random
numbers, and non-determinism.

\subsubsection{State}

In general, effects are described algebraically in terms of the operations
they support. In the case of \texttt{State}, the supported effects are reading
the state (\texttt{Get}) and writing the state (\texttt{Put}).

\begin{SaveVerbatim}{statealg}

data State : Effect where
     Get :      State a a a
     Put : b -> State a b ()

\end{SaveVerbatim}
\useverb{statealg}

\noindent
The resource associated with a state corresponds to the state itself. So,
the \texttt{Get} operation leaves this state intact (with a resource type
\texttt{a} on entry and exit) but the \texttt{Put} operation may update this
state (with a resource type \texttt{a} on entry and \texttt{b} on exit)
--- that is, a \texttt{Put} may update the type of the stored value.
We can implement a handler for this effect, for all contexts \texttt{m},
as follows:

\begin{SaveVerbatim}{statehandle}

instance Handler State m where
     handle st Get     k = k st st
     handle st (Put n) k = k n ()

\end{SaveVerbatim}
\useverb{statehandle}

\noindent
When running \texttt{Get}, the handler passes the current state to the
continuation as both the new resource value (the first argument of the
continuation \texttt{k}) as well as the return value of the computation (the
second argument of the continuation). When running \texttt{Put}, the new state
is passed to the continuation as the new resource value.

We then convert the algebraic effect \texttt{State} to a concrete
effect usable in an \Eff{} program using the \texttt{STATE} function, to which
we provide the initial state type as follows:

\begin{SaveVerbatim}{statepromote}

STATE : Type -> EFFECT
STATE t = MkEff t State

\end{SaveVerbatim}
\useverb{statepromote}

\noindent
We adopt the convention that algebraic effects, of type \texttt{Effect},
have an initial upper case letter. Concrete effects, of type \texttt{EFFECT},
are correspondingly in all upper case.

Algebraic effects are promoted to \Eff{} programs with concrete effects
by using the
\texttt{mkEffect} function. We will postpone giving the type of
\texttt{mkEffect} until Section \ref{sect:mkeffect} --- for now,
it suffices to know that it converts an
\texttt{Effect} to an effectful program. To create the \texttt{get} and
\texttt{put} functions, for example:

\begin{SaveVerbatim}{getdef}

get : Eff m [STATE x] x
get = mkEffect Get 

\end{SaveVerbatim}
\useverb{getdef}

\begin{SaveVerbatim}{putdef}
put : x -> Eff m [STATE x] ()
put val = mkEffect (Put val)

\end{SaveVerbatim}
\useverb{putdef}

\noindent
We may also find it useful to mutate the \remph{type} of a state, considering
that states may themselves have dependent types (we may, for example, add
an element to a vector in a state). The \texttt{Put} constructor supports this,
so we can implement \texttt{putM} to update the state's type:

\begin{SaveVerbatim}{putmdef}

putM : y -> EffM m [STATE x] [STATE y] ()
putM val = mkEffect (Put val)

\end{SaveVerbatim}
\useverb{putmdef}

\noindent
Finally, it may be useful to combine \texttt{get} and \texttt{put} in a single
update:

\begin{SaveVerbatim}{update}

update : (x -> x) -> Eff m [STATE x] ()
update f = do val <- get; put (f val) 
\end{SaveVerbatim}
\useverb{update}

\subsubsection{Console I/O}

We consider a simplified version of console I/O which supports reading
and writing strings to and from the console. There is no associated resource,
although in an alternative implementation we may associate it with an abstract
world state, or a pair of file handles for \texttt{stdin}/\texttt{stdout}.
Algebraically:
%we describe console I/O as follows:

\begin{SaveVerbatim}{stdioeff}

data StdIO : Effect where
     PutStr : String -> StdIO () () ()
     GetStr : StdIO () () String

STDIO : EFFECT
STDIO = MkEff () StdIO

\end{SaveVerbatim}
\useverb{stdioeff}

\noindent
The obvious way to handle \texttt{StdIO} is by translating via the \texttt{IO}
monad, which is implemented straightforwardly as follows:

\begin{SaveVerbatim}{stdiohandle}

instance Handler StdIO IO where
    handle () (PutStr s) k = do putStr s; k () ()
    handle () GetStr     k = do x <- getLine; k () x 

\end{SaveVerbatim}
\useverb{stdiohandle}

\noindent
Unlike the \texttt{State} effect, for which the handler worked in \remph{all}
contexts, this handler only applies to effectful programs run in an \texttt{IO}
context. We can implement alternative handlers, and indeed there is no
reason that effectful programs in \texttt{StdIO} must be evaluated in a monadic
context. For example, we can define I/O stream functions:
%and an associated handler:

\begin{SaveVerbatim}{iostream}

data IOStream a 
   = MkStream (List String -> (a, List String))

instance Handler StdIO IOStream where
    ...
\end{SaveVerbatim}
\useverb{iostream}

\noindent
A handler for \texttt{StdIO} in \texttt{IOStream} context generates a function
from a list of strings (the input text) to a value and the output text. We
can build a pure function which simulates real console I/O:

\begin{SaveVerbatim}{mkstrfunty}

mkStrFn : Env IOStream xs -> Eff IOStream xs a -> 
          List String -> (a, List String)
\end{SaveVerbatim}
\useverb{mkstrfunty}

\begin{SaveVerbatim}{mkstrfun}
mkStrFn {a} env p input = case mkStrFn' of
                               MkStream f => f input
  where injStream : a -> IOStream a
        injStream v = MkStream (\x => (v, []))
        mkStrFn' : IOStream a
        mkStrFn' = runWith injStream env p

\end{SaveVerbatim}
\useverb{mkstrfun}

\noindent
To illustrate this, we write a simple console I/O program which runs in
any context which has a handler for \texttt{StdIO}:

\begin{SaveVerbatim}{ioname}

name : Handler StdIO e => Eff e [STDIO] ()
name = do putStr "Name? "
          n <- getStr
          putStrLn ("Hello " ++ show n)

\end{SaveVerbatim}
\useverb{ioname}

\noindent
Using \texttt{mkStrFn}, we can run this as a pure function which uses a list
of strings as its input, and gives a list of strings as its output. We can
evaluate this at the \Idris{} prompt:

\begin{SaveVerbatim}{mkstrfunrun}

*name> show $ mkStrFn [()] name ["Edwin"]
((), ["Name?" , "Hello Edwin\n"]) 

\end{SaveVerbatim}
\useverb{mkstrfunrun}

\noindent

\begin{SaveVerbatim}{blehlatex}
$
\end{SaveVerbatim}

\noindent
This suggests that alternative, pure, handlers for console I/O,
or any I/O effect, can be used for unit testing and reasoning about I/O programs
without executing any real I/O.
%and possibly even proving theorems about them.

\subsubsection{Exceptions}

The exception effect supports only one operation, \texttt{Raise}.
Exceptions are parameterised over an error type \texttt{e}, so \texttt{Raise}
takes a single argument to represent the error. The associated resource is
of unit type, and since raising an exception causes computation to abort, 
raising an exception can return a value of any type.
%, since it will not be used.

\begin{SaveVerbatim}{exctype}

data Exception : Type -> Effect where
     Raise : e -> Exception e () () b 

EXCEPTION : Type -> EFFECT
EXCEPTION e = MkEff () (Exception e) 

\end{SaveVerbatim}
\useverb{exctype}

\noindent
The semantics of \texttt{Raise} is to abort computation, therefore handlers
of exception effects do not call the continuation \texttt{k}. In any case, 
this should be impossible since passing the result to the continuation would
require the ability to invent a value in any arbitrary type \texttt{b}!
The simplest handler runs in a \texttt{Maybe} context:

\begin{SaveVerbatim}{excmaybe}

instance Handler (Exception a) Maybe where
     handle _ (Raise err) k = Nothing

\end{SaveVerbatim}
\useverb{excmaybe}

\noindent
Exceptions can be handled in any context which supports some representation of
failed computations. In an \texttt{Either e} context, for example, we can
use \texttt{Left} to represent the error case:

\begin{SaveVerbatim}{exceither}

instance Handler (Exception e) (Either e) where
     handle _ (Raise err) k = Left err

\end{SaveVerbatim}
\useverb{exceither}

\noindent
Given that we can raise exceptions in an \Eff{} program, it is also useful to be
able to catch them. The \texttt{catch} operation runs a possibly failing
computation \texttt{comp} in some context \texttt{m}:
%, running \texttt{handler} if the computation fails:

\begin{SaveVerbatim}{catch}

catch : Catchable m err =>
        (comp : EffM m xs xs' a) -> 
        (handler : err -> EffM m xs xs' a) ->
        EffM m xs xs' a

\end{SaveVerbatim}
\useverb{catch}

\noindent
Using \texttt{catch} requires that the computation context \texttt{m} supports
throwing and catching of errors, using the \texttt{Catchable} class implemented
as part of the \Idris{} library:

\begin{SaveVerbatim}{catchable}

class Catchable (m : Type -> Type) t where
    throw : t -> m a
    catch : m a -> (t -> m a) -> m a

\end{SaveVerbatim}
\useverb{catchable}

\noindent
This is implemented directly as part of the \Eff{} DSL.
%--- any effectful
%computation may fail, and therefore implement a handler in a \texttt{Catchable}
%context after all. 
We have a simple instance for \texttt{Maybe}, provided that
the error thrown is the unit error:

\begin{SaveVerbatim}{catchablemaybe}

instance Catchable Maybe () where
    throw () = Nothing
    catch Nothing  h = h ()
    catch (Just x) h = Just x

\end{SaveVerbatim}
\useverb{catchablemaybe}

\noindent
The instance for \texttt{Either e} is similar, except that any type may be
thrown as an error:

\begin{SaveVerbatim}{catchableeither}

instance Catchable (Either e) e where
    throw x = Left x
    catch (Left err) h = h err
    catch (Right x)  h = (Right x)

\end{SaveVerbatim}
\useverb{catchableeither}

\subsubsection{Random numbers}

Random number generation can be implemented as an effect, with the resource
tracking the \remph{seed} from which the next number will be generated.
The \texttt{Random} effect supports one operation, \texttt{getRandom}, which
requires an \texttt{Int} resource and returns the next number:

\begin{SaveVerbatim}{randomeff}

data Random : Type -> Type -> Type -> Type where
     GetRandom : Random Int Int Int

RND : EFFECT
RND = MkEff Int Random

\end{SaveVerbatim}
\useverb{randomeff}

\noindent
Handling random number generation shows that it is a state effect in
disguise, where the effect updates the seed.
This is a simple linear congruential pseudo-random number generator:

\begin{SaveVerbatim}{randomhandle}

instance Handler Random m where
    handle seed GetRandom k
         = let seed' = 1664525 * seed + 1013904223 in
               k seed' seed'

\end{SaveVerbatim}
\useverb{randomhandle}

\noindent
Alternative handlers could use a different, possibly more secure approach.
In any case, we can implement a top level function which returns a random
number between a lower and upper bound as follows:

\begin{SaveVerbatim}{rndint}

rndInt : Int -> Int -> Eff m [RND] Int
rndInt lower upper 
    = do v <- mkEffect GetRandom 
         return (v `mod` (upper - lower) + lower)

\end{SaveVerbatim}
\useverb{rndint}

\subsubsection{Resource management: Files}

Instead of implementing all I/O operations as a single effect, as with the
\texttt{IO} monad, we can separate operations into more fine-grained effects.
After Console I/O, another I/O related effect which we can handle separately
is file management. Here, we can take advantage of the \remph{resource}
associated with an effect, and the fact that resource types are \remph{mutable},
by associating the file handling effect with an individual file, parameterised
by its current state (i.e. closed, open for reading, or open for writing).
File handles are represented as follows, where \texttt{File} is a primitive
file handle provided by the \Idris{} library:

\begin{SaveVerbatim}{fileeff}

data Mode = Read | Write

data OpenFile : Mode -> Type where
     FH : File -> OpenFile m

\end{SaveVerbatim}
\useverb{fileeff}

\noindent
When we declare the \texttt{FileIO} algebraic effect type, we express in
the resource transitions how each effect changes the state of the resource:

\begin{SaveVerbatim}{fileio}

data FileIO : Effect where
     Open  : String -> (m : Mode) -> 
             FileIO () (OpenFile m) ()
     Close : FileIO (OpenFile m) () ()
\end{SaveVerbatim}
\useverb{fileio}

\begin{SaveVerbatim}{fileio2}
     ReadLine  : FileIO (OpenFile Read)  
                        (OpenFile Read) String
     WriteLine : String -> 
                 FileIO (OpenFile Write) 
                        (OpenFile Write) ()
     EOF       : FileIO (OpenFile Read)  
                        (OpenFile Read) Bool

\end{SaveVerbatim}
\useverb{fileio2}

\noindent
We can see from this declaration that opening a file moves from an empty
resource to a file open for a specific purpose and that closing a file removes
the file resource. Only files which are open for reading may be read, or tested
for end of file, and only files open for writing may be written to. Any
violation of this resource access protocol will result in a type error.
In general, we can use the effect signature of a function to manage resources
safely, subsuming the resource management DSL we have previously
implemented~\cite{bradyresource}.

The \texttt{FILE\_IO} effect is parameterised over the current state of a
file resource with which it is associated:

\begin{SaveVerbatim}{fileres}

FILE_IO : Type -> EFFECT

\end{SaveVerbatim}
\useverb{fileres}

\noindent
The type of \texttt{open} expresses that the resource changes from a unit
to an open file:

\begin{SaveVerbatim}{openeff}

open : String -> (m : Mode) -> 
       EffM IO [FILE_IO ()] [FILE_IO (OpenFile m)] ()

\end{SaveVerbatim}
\useverb{openeff}

\noindent
Note that opening a file may fail --- we will deal with exceptional behaviour
shortly. 
Using \texttt{EffM}, we have expressed that opening a file causes a change in the
resource state.
It is then only possible to close a file if there is an open file
available:

\begin{SaveVerbatim}{closeeff}

close : EffM IO [FILE_IO (OpenFile m)] [FILE_IO ()] ()

\end{SaveVerbatim}
\useverb{closeeff}

\noindent
Reading is only possible from a file opened for reading:

\begin{SaveVerbatim}{readeff}

readLine : Eff IO [FILE_IO (OpenFile Read)] String

\end{SaveVerbatim}
\useverb{readeff}

\noindent
As with \texttt{STATE}, we can use labelled resources if we require more than
one file handle at a time. We have handlers for \texttt{FileIO} for the
\texttt{IO} context, which does not handle exceptions (in which case failing
to open a file is a fatal run-time error), and an \texttt{IOExcept e}
context which is \texttt{IO} augmented with exceptions of type \texttt{e}
and an instance of the \texttt{Catchable} class:

\begin{SaveVerbatim}{ioexcept}

data IOExcept err a
ioe_lift : IO a -> IOExcept err a

instance Catchable IOExcept err

\end{SaveVerbatim}
\useverb{ioexcept}

\noindent
Assuming we are in a state where we have a file handle available and open
for reading, we can read the contents of a file into a list of strings:

\noindent
\begin{SaveVerbatim}{readfile}

 readLines : Eff (IOExcept String) 
                [FILE_IO (OpenFile Read)] (List String)
 readLines = readAcc [] where
   readAcc : List String -> 
             Eff (IOExcept String) 
                 [FILE_IO (OpenFile Read)] (List String)
   readAcc acc = do e <- eof
                    if (not e)
                       then do str <- readLine
                               readAcc (str :: acc)
                       else return (reverse acc)

\end{SaveVerbatim}
\useverb{readfile}

\noindent
To read a file, given a file path, into a list of strings, reporting an
error where necessary, we can write the following \texttt{readFile}
function. We add \texttt{STDIO} to the list of effects so that we can report
an error to the console if opening the file fails:

\noindent
\begin{SaveVerbatim}{readfileproto}

 readFile : String -> Eff (IOExcept String)
                [FILE_IO (), STDIO] (List String)
 readFile path = catch (do open path Read
                           lines <- readLines
                           close
                           return lines)
         (\err => do putStrLn ("Failed " ++ err)
                     return [])

\end{SaveVerbatim}
\useverb{readfileproto}

\noindent
The effect type of \texttt{readFile} means that we must begin \remph{and} end
with no open file. This means that omitting the \texttt{close} would result in
a compile time type error. It would also be a type error to try to invoke
\texttt{readLines} before the file was open, or if the file was opened for
writing instead.

\subsubsection{Non-determinism}

Following~\cite{Bauer},
non-determinism can be implemented as an effect \texttt{Selection},
in which a \texttt{Select} operation chooses one value non-deterministically
from a list of possible values:

\begin{SaveVerbatim}{selecteff}

data Selection : Effect where
     Select : List a -> Selection () () a

\end{SaveVerbatim}
\useverb{selecteff}

\noindent
We can handle this effect in a \texttt{Maybe} context, trying every choice
in a list given to \texttt{Select} until the computation succeeds:

\begin{SaveVerbatim}{selectmaybe}

instance Handler Selection Maybe where
     handle _ (Select xs) k = tryAll xs where
         tryAll [] = Nothing
         tryAll (x :: xs) = case k () x of
                                 Nothing => tryAll xs
                                 Just v => Just v

\end{SaveVerbatim}
\useverb{selectmaybe}

\noindent
The handler for \texttt{Maybe} produces one result if it exists, effectively
performing a depth first search of all of the values passed to \texttt{Select}.
Note in particular that the handler runs the continuation for every element
of the list until the result of running the continuation succeeds.

Alternatively, we can find \remph{every} possible result by handling selection
in a \texttt{List} context:

\begin{SaveVerbatim}{selectlist}

instance Handler Selection List where
     handle r (Select xs) k = concatMap (k r) xs

\end{SaveVerbatim}
\useverb{selectlist}

\noindent
We can use the \texttt{Selection} effect to implement search problems by
non-deterministically choosing from a list of candidate solutions. For example,
a solution to the n-Queens problem can be implemented as follows.
First, we write a function which checks whether a point on a chess board
attacks another if occupied by a Queen:

\begin{SaveVerbatim}{noattack}

no_attack : (Int, Int) -> (Int, Int) -> Bool
no_attack (x, y) (x', y')
  = x /= x' && y /= y' && abs (x - x') /= abs (y - y')

\end{SaveVerbatim}
\useverb{noattack}

\noindent
Then, given a column and a list of Queen positions, we find the rows on which
a Queen may safely be placed in that column using a list comprehension:

\begin{SaveVerbatim}{rowsin}

rowsIn : Int -> List (Int, Int) -> List Int
rowsIn col qs 
   = [ x | x <- [1..8], all (no_attack (x, col)) qs ]

\end{SaveVerbatim}
\useverb{rowsin}

\noindent
Finally, we compute a solution by accumulating a set of Queen positions,
column by column, non-deterministically choosing a position for a Queen in
each column.

\begin{SaveVerbatim}{nqueens}

addQueens : Int -> List (Int, Int) -> 
            Eff m [SELECT] (List (Int, Int))
addQueens 0   qs = return qs
addQueens col qs 
   = do row <- select (rowsIn col qs)
        addQueens (col - 1) ((row, col) :: qs)

\end{SaveVerbatim}
\useverb{nqueens}

\noindent
We can run this in \texttt{Maybe} context, to retrieve one solution, or in
\texttt{List} context, to retrieve all solutions. In a
\texttt{Maybe} context, for example, we can define:

\begin{SaveVerbatim}{getqueens}

getQueens : Maybe (List (Int, Int))
getQueens = run [()] (addQueens 8 [])

\end{SaveVerbatim}
\useverb{getqueens}

\noindent
Then to find the first solution, we run \texttt{getQueens} at the %\Idris{}
REPL:

\begin{SaveVerbatim}{queenmaybe}

*Queens> show getQueens
"Just [(4, 1), (2, 2), (7, 3), (3, 4), 
       (6, 5), (8, 6), (5, 7), (1, 8)]" : String

\end{SaveVerbatim}
\useverb{queenmaybe}

\noindent
\textbf{Remark:} It is important to note that when combining \texttt{SELECT}
with other effects, the values of other resources are reset at the beginning
of each \texttt{select} branch. This means, at least in the current implementation,
that state cannot be shared between branches. While perhaps not so important
for selection, this may be a problem for other control effects such as 
co-operative multithreading, for which we may need a more flexible handler if
we wish to deal with shared state. We will deal with this issue in future work.

\subsection{Effect Polymorphism}

Since \Eff{} is implemented as an embedding in a host language, we can exploit
features of the host language. The means that we can write higher order
functions, and functions which are polymorphic in their effects. For example,
a \texttt{mapEff} function can be implemented corresponding to \texttt{fmap}
for functors:

\noindent
\begin{SaveVerbatim}{mapeff}

mapEff : Applicative m => 
         (a -> Eff m xs b) -> List a -> Eff m xs (List b)
mapEff f []        = pure [] 
mapEff f (x :: xs) = [| f x :: mapEff f xs |]

\end{SaveVerbatim}
\useverb{mapeff}

\noindent
This applies an effectful function across a list, provided that we are in
an applicative context, and that the effectful function supports the correct
set of effects.
