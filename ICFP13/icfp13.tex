\documentclass[preprint]{sigplanconf}
%\documentclass[orivec,dvips,10pt]{llncs}

%\conferenceinfo{ICFP'10,} {September 27--29, 2010, Baltimore, Maryland, USA.}
%\CopyrightYear{2010}
%\copyrightdata{978-1-60558-794-3/10/09}

\usepackage[draft]{comments}
%\usepackage[final]{comments}
% \newcommand{\comment}[2]{[#1: #2]}
\newcommand{\khcomment}[1]{\comment{KH}{#1}}
\newcommand{\ebcomment}[1]{\comment{EB}{#1}}

\usepackage{epsfig}
\usepackage{path}
\usepackage{url}
\usepackage{amsmath,amssymb} 
\usepackage{fancyvrb}

\newenvironment{template}{\sffamily}

\usepackage{graphics,epsfig}
\usepackage{stmaryrd}

\input{./macros.ltx}
\input{./library.ltx}

\NatPackage
\FinPackage

\newcounter{per}
\setcounter{per}{1}

\newcommand{\Idris}{\textsc{Idris}}
\newcommand{\Eff}{\texttt{Effects}}

\newcommand{\pebox}[2]{\vspace*{0.1cm}\noindent
\begin{center}
\fbox{
\begin{minipage}{7.5cm}\textbf{#1:} #2\addtocounter{per}{1}
\end{minipage}}
\end{center}
\vspace*{0.1cm}
}

\newcommand{\perule}[1]{\vspace*{0.1cm}\noindent
\begin{center}
\fbox{
\begin{minipage}{7.5cm}\textbf{Rule \theper:} #1\addtocounter{per}{1}
\end{minipage}}
\end{center}
\vspace*{0.1cm}
}

\newcommand{\mysubsubsection}[1]{
\noindent
\textbf{#1}
}
\newcommand{\hdecl}[1]{\texttt{#1}}

\begin{document}

\title{Programming and Reasoning with Algebraic Effects}
\subtitle{Or: $\Pi$, the Ultimate Monad Transformer}
\authorinfo{Edwin C. Brady}
{School of Computer Science, 
University of St Andrews, St Andrews, Scotland.}
{Email: ecb10@st-andrews.ac.uk}

\maketitle

\begin{abstract}
One often cited benefit of pure functional programming is that pure code is
easier to test and reason about, both formally and informally. However, in
order to be useful, programs must interact with the outside world in some way
--- real programs have state, open files, send data over a network, draw
pictures, and so on. Haskell solves this problem using \remph{monads} to capture
details of possibly side effecting computations --- it provides monads for
capturing State, I/O, exceptions, non-determinism, libraries for practical
purposes such as CGI and parsing, and many others, as well as \remph{monad
transformers} for combining multiple effects.

Unfortunately, useful as monads are, they do not compose very well. Monad
transformers can quickly become unwieldy when there are lots of effects to
manage, leading to a temptation in larger programs to combine everything into
one coarse-grained state and exception monad. In this paper I describe an
alternative approach based on handling ~\remph{algebraic effects}, implemented
in the Idris programming language. I show how to describe side effecting
computations, how to write programs which compose multiple fine-grained
effects, and how, using dependent types, we can use this approach to reason
about states in effectful programs.
\end{abstract}

%\category{D.3.2}{Programming Languages}{Language
%  Classifications}[Applicative (functional) Languages]
%\category{D.3.4}{Programming Languages}{Processors}[Compilers]
%\terms{Languages, Verification, Performance}
%\keywords{Dependent Types, Partial Evaluation}

\section{Introduction}

%Pure functional programming languages with \remph{dependent types} such as
%\Idris{}~\cite{idristutorial}, Agda~\cite{norell2009dependently} and
%Coq~\cite{Bertot2004} offer a programmer can both to write programs and verify
%specifications within the same framework. Thanks to the Curry-Howard
%correspondence, a program which type checks can be viewed as a proof that it
%conforms to the specification given by its type. Dependent types allow
%more precise types, and therefore more precise specifications.

Pure functions do not have side effects, but real applications do:
they may have state, communicate across a network, read and write files, 
or interact with users, among many other things. Furthermore, real systems
may fail --- data may be corrupted or untrusted. Pure functional programming
languages such as Haskell~\cite{Haskell98} manage such problems using
\remph{monads}~\cite{Wadler1995}, and allow multiple effects to be combined
using a stack of \remph{monad transformers}~\cite{Liang1995}.

Monad transformers are an effective tool for structuring larger Haskell
applications. A simple application using a Read-Execute-Print loop,
for example, may have some global state and perform console I/O, and hence
be built with an \texttt{IO} monad transformed into a state monad
using the \texttt{StateT} transformer. However, there are some difficulties
with building applications in this way. Two of the most important
are that the order in which transformers are applied matters (that is, 
transformers do not commute easily), and that it is difficult to invoke a
function which uses a subset of the transformers in the stack.
To illustrate these problem, consider an evaluator, in Haskell, for a 
simple expression language:

\begin{SaveVerbatim}{exprlang1}

data Expr = Val Int | Add Expr Expr

eval : Expr -> Int
eval (Val i) = i
eval (Add x y) = eval x + eval y

\end{SaveVerbatim}
\useverb{exprlang1}

\noindent
If we extend this language with variables, we need to extend the evaluator
with an environment, and deal with possible failure if a variable is undefined
(we omit the \texttt{Val} and \texttt{Add} cases):

\begin{SaveVerbatim}{exprlang2}

data Expr = Val Int | Add Expr Expr | Var String

type Env = [(String, Int)]

eval :: Expr -> ReaderT Env Maybe Int
eval (Var x) = do env <- ask
                  val <- lift (lookup x env)
                  return val

\end{SaveVerbatim}
\useverb{exprlang2}

\noindent
Here, the \texttt{Maybe} monad captures possible failure, and is transformed
into a reader monad using the \texttt{ReaderT} transformer to store the
environment, which is retrieved using \texttt{ask}. The \texttt{lift} operation
allows functions in the inner \texttt{Maybe} monad to be called.  
We can extend the language further, with random number generation:

\begin{SaveVerbatim}{exprlang3}

data Expr = Val Int | Add Expr Expr | Var String
          | Random Int

eval :: Expr -> RandT (ReaderT Env Maybe) Int
eval (Var x) = do env <- lift ask
                  val <- lift (lift (lookup x env))
                  return val
eval (Random x) = do val <- getRandomR (0, x)
                     return val

\end{SaveVerbatim}
\useverb{exprlang3}

\noindent
We have added another transformer to the stack, \texttt{RandT}, and added
\texttt{lift} where necessary to access the appropriate monads in the stack.
We have been able to build this interpreter from reusable components --- the
\texttt{Maybe} monad and \texttt{ReaderT} and \texttt{RandT} transformers ---
which is clearly a good thing. One small problem, however, is that the use of
\texttt{lift} is a little noisy, and will only get worse if we add more
monads to the stack, such as \texttt{IO}, though implementations of 
\texttt{lift} can be automated~\cite{Jaskelioff2009}.
Bigger problems occur if we need
to permute the order of the transformers, or invoke a function which uses a
subset, for example:

\begin{SaveVerbatim}{mtransprobs}

permute :: ReaderT Env (RandT Maybe) a -> 
           RandT (ReaderT Env Maybe) a
dropReader :: RandT Maybe a -> 
              RandT (ReaderT Env Maybe) a

\end{SaveVerbatim}
\useverb{mtransprobs}

\noindent
These problems mean that, in general, there is little motivation for separating
effects, and a temptation to build an application around one general purpose
monad capturing all of an application's state and exception handling needs.
%
It would be desirable, on the other hand, to separate effects into specific
components such as console I/O, file and network handling and operating system
interaction, for the same reason that it is desirable to separate the pure
fragments of a program from the impure fragments using the \texttt{IO} monad.
That is, the program's type would give more precise information about what the
program is supposed to do, making the program easier to reason about and
reducing the possibility of errors.

In this paper, I present an alternative approach to combining effects in
a pure functional programming language, based on handlers of
\remph{algebraic effects}~\cite{Bauer}, and
implemented directly as a domain specific language embedded
in a dependently typed host, \Idris{}~\cite{Brady2013,idristutorial}.

\subsection{Contributions}

This paper takes as its hypothesis that algebraic effects provide a cleaner,
more composable and more flexible notation for programming with side effects
than monad transformers. Although they are not equivalent in power --- monads
and monad transformers can express more concepts --- many common effectful
computations are captured. 
The main contribution of this paper is a notation for describing and combining
side effects using \Idris{}. More specifically:

%, the paper makes the following contributions:

\begin{itemize}
\item An Embedded Domain Specific Language (DSL), \Eff{}, for capturing \emph{algebraic
effects}, in such a way that they are easily composable, and translatable to a
variety of underlying contexts using \emph{effect handlers}.
% \item An extension of the DSL for tracking the \emph{state} of resources
% associated with effects, such as whether a file handle is open or a network
% socket is ready to receive data.
\item A collection of example effects (State, Exceptions,
File and Console I/O, random number generation and
non-determinism) and their handlers. I show how alternative handlers
can be used to evaluate effects in different contexts. In particular, we can
use an alternative handler to run interactive programs in a pure context.
%implement unit tests for (and possibly reason about) interactive programs by
%handling the effect in the context of input and output streams.
\item I give example programs which combine effects, including a 
an interpreter for an imperative language with mutable variables, to
illustrate how effectful applications may be structured.
\end{itemize}

\noindent
The \Eff{} DSL makes essential use of \emph{dependent types}, firstly to verify
that a specific effect is available to an effectful program using simple
automated theorem proving, and secondly to track the state of a resource by
updating its type during program execution. In this way, we can use the \Eff{}
DSL to verify implementations of resource usage protocols.

I describe how to \remph{use} \Eff{} in Section \ref{sect:effdsl},
how it is implemented in Section \ref{sect:effimpl}, and give a larger example
in Section \ref{sect:impint}. It is distributed with
\Idris{}\footnote{\url{http://idris-lang.org/}}. All of the examples in this
paper are available online at \url{http://idris-lang.org/effects}.



%[Idea: We can use the ability to implement different handlers to make
%unit tests for interactive programs. Worth doing? Also, possibly, reasoning
%about effectful programs.]

%\subsection{Outline}

% TODO: Drop this into the narrative above...

%This paper is structured as follows: In Section \ref{sect:idris} I briefly
%introduce the \Idris{} programming language, in particular the distinctive
%features we will use in the \Eff{} implementation; Section \ref{sect:effectsfp}
%motivates the problem of effect handling by describing an effectful evaluator
%implemented in \Idris{}; in Section \ref{sect:effdsl} I describe the \Eff{} DSL,
%showing how to \remph{use} defined effects, and how to define new effects;
%Section \ref{sect:effimpl} covers the implementation of \Eff{}; in Section
%\ref{sect:impint} I give a larger example, an interpreter for an imperative
%language implemented using \Eff{}, then finally I discuss related work and
%conclude in Sections \ref{sect:related} and \ref{sect:conclusion}.


%Possible structure:

%\begin{itemize}
%\item Monad transformers and why they don't compose nicely
%\item Motivating example in \Idris{}
%\item How to use it, how to create effects
%\item How it works
%\item Bigger example (CGI?)
%\end{itemize}



\section{Programming in \Idris{}}

\label{sect:idris}

\Idris{} is a pure functional programming language with \remph{dependent
types}. It is eagerly evaluated by default (though programs may be annotated
for laziness) and compiled.  Like modern Haskell, it supports multi-parameter
type classes, though these are intended primarily for overloading. The syntax
is based heavily on Haskell, with the exception that a single colon is used for
a type declaration and a double colon for list construction, emphasising the
importance of types.  High level syntactic sugar includes support for
\texttt{where} clauses, \texttt{do}-notation, monad comprehensions and idiom
brackets~\cite{McBride2007}. In this section, we briefly discuss the language
features we will use in this paper, in particular support for tactic
based theorem proving and implicit syntax for assisting with 
embedded domain specific language construction. A detailed tutorial
is available elsewhere~\cite{idristutorial}.

%Very brief intro, needs to cover \texttt{Vect}, membership predicates, simple
%theorem proving and tactic reflection, default implicit arguments and implicit
%conversions.  Also (maybe) named classes to show how multiple handlers can be
%created.  A full tutorial is available elsewhere~\cite{idristutorial}.

\subsection{Types and functions}

Data types are declared in a similar way to Haskell data types, with a similar
syntax. Natural numbers and lists, for example, are declared as follows in the
library:

\begin{SaveVerbatim}{natlist}

data Nat    = O   | S Nat           
data List a = Nil | (::) a (List a) 

\end{SaveVerbatim}
\useverb{natlist}

A standard example of a \remph{dependent} type is the type of ``lists with
length'', conventionally called ``vectors'' in the dependently typed
programming literature. In \Idris{}, vectors are declared as follows:

\begin{SaveVerbatim}{vect}

data Vect : Type -> Nat -> Type where
   Nil  : Vect a O
   (::) : a -> Vect a k -> Vect a (S k)

\end{SaveVerbatim}
\useverb{vect}

\noindent
Note that this uses the same constructor names as for \texttt{List}. Ad-hoc name
overloading such as this is accepted by \Idris{}, provided that the names are
declared in different namespaces (in practice, normally in different modules)
so that the names are different internally. Namespace resolution can be made
explicitly (e.g. \texttt{List.Nil} or \texttt{Vect.Nil}) or more commonly
by type.

Syntactic sugar is provided for lists, with the notation \texttt{[a,b]}
expanding to \texttt{a :: b :: Nil}. Name overloading means that this
sugar applies to \remph{any} type using the constructors \texttt{Nil}
and \texttt{(::)}.

Functions are defined by pattern matching. A function over a dependent type
naturally expresses in its type some invariants properties of that function.
For example, appending two vectors expresses the invariant that the output
length is the sum of the input lengths:

\begin{SaveVerbatim}{vappend}

(++) : Vect a n -> Vect a m -> Vect a (n + m)
Nil       ++ ys = ys
(x :: xs) ++ ys = x :: xs ++ ys

\end{SaveVerbatim}
\useverb{vappend}

\noindent
Expressions may be evaluated at the \Idris{} prompt, which provides a 
read-eval-print loop with GHCI-style commands. For example:

\begin{SaveVerbatim}{irepl}

Idris> show ([1, 2, 3] ++ [4, 5, 6])
"[1, 2, 3, 4, 5, 6]" : String

\end{SaveVerbatim}
\useverb{irepl}

\subsection{Tactic-based theorem proving}

\Idris{} supports Coq-style tactic based theorem proving~\cite{Bertot2004}
although presently the range of available tactics is limited. Tactic scripts
may be given using the \texttt{:p} command at the \Idris{} prompt, or directly
in a program. For example, the following program proves that \texttt{n + O = n}
for all \texttt{n}, using a combination of pattern matching and tactics:

\begin{SaveVerbatim}{plusnOprf}

plus_nO : (n : Nat) -> n + O = n
plus_nO O     = refl
plus_nO (S k) = let indH = plus_nO k in ?pnO_Scase

plus_theorem.pnO_Scase = proof {
  compute; intros;
  rewrite indH; trivial;
}

\end{SaveVerbatim}
\useverb{plusnOprf}

\noindent
The notation \texttt{?pnO\_Scase} introduces a \remph{metavariable} with the
proof postponed until later. This proof is given using a tactic script introduced
by the keyword \texttt{proof}.

\subsection{Generating tactics}

It is also possible to write programs which \remph{compute} tactics, introducing
limited support for automated proof construction. Tactics are represented
as follows:

\begin{SaveVerbatim}{tactic}

data Tactic 
     = Try Tactic Tactic | Refine TTName
     | Seq Tactic Tactic | Trivial 
     | GoalType TTName Tactic | Exact TT | Solve

\end{SaveVerbatim}
\useverb{tactic}

\noindent
This includes the primitive tactics \texttt{Refine} (which attempts to make
progress by applying a function), \texttt{Trivial} (which attempts to solve
a goal by reflexivity or by finding a definition in the context), and
\texttt{Exact} (which attempts to solve a goal by providing an exact proof).
There are combinators \texttt{Try} (which applies its first argument, and if that
fails applies its second argument), \texttt{Seq} (which applies two
tactics in sequence) and \texttt{GoalType} (which applies a tactic only if the
goal is a specific type). Finally, \texttt{Solve} is used to close a goal
when a sub-proof is complete --- in \texttt{proof} scripts such as that used
to prove \texttt{plus\_nO} the \texttt{Solve} tactic is applied automatically
to close all subgoals.

In proof mode, the tactic \texttt{reflect t} applies the constructed tactic
\texttt{t}, provided that \texttt{t} has type \texttt{Tactic}. Full details
of tactic based theorem proving, including details of the \texttt{Solve}
mechanism, are described elsewhere~\cite{Brady2013} and
beyond the scope of this paper, although
we will be using basic tactic construction to construct simple list membership
proofs automatically.

\subsection{Implicit syntax}

\subsubsection{Arguments}

Let us inspect the type of \texttt{(++)} more closely:

\begin{SaveVerbatim}{appendtype}

(++) : Vect a n -> Vect a m -> Vect a (n + m)

\end{SaveVerbatim}
\useverb{appendtype}

\noindent
It takes two arguments, being the lists to append. However, there are also
three names, \texttt{a}, \texttt{n} and \texttt{m} which are not bound
explicitly. These are \remph{implicit arguments} to \texttt{(++)}. The
type could also be written as:

\begin{SaveVerbatim}{appendtypeimpl}

(++) : {a : Type} -> {n : Nat} -> {m : Nat} ->
       Vect a n -> Vect a m -> Vect a (n + m)

\end{SaveVerbatim}
\useverb{appendtypeimpl}

\noindent
This gives bindings for \texttt{a}, \texttt{n} and \texttt{m}. Implicit
arguments, given in braces \texttt{\{\}} in the type signature, are not
given in applications of \texttt{(++)}; their values can be inferred 
by unification from the types of the two explicit arguments. Implicit arguments
may be given explicitly in applications using the syntax \texttt{\{a=value\}},
for example:

\begin{SaveVerbatim}{appendexpl}

(++) {a=Int} {n=2} {m=3} [1,2] [3,4,5]

\end{SaveVerbatim}
\useverb{appendexpl}

\noindent
In this context, \texttt{\{a\}} is a shorthand for \texttt{\{a=a\}}. In
general, implicit arguments in applications are solved by unification. However,
unification may not be strong enough.
In other situations, it may be possible to infer arguments not by unification
but by providing a tactic for automatic proof construction.
For example, the following definition of \texttt{head} which requires a proof
that the list is non-empty

\begin{SaveVerbatim}{safehead}

isCons : List a -> Bool
isCons [] = False
isCons (x :: xs) = True

head : (xs : List a) -> (isCons xs = True) -> a
head (x :: xs) _ = x

\end{SaveVerbatim}
\useverb{safehead} 

\noindent
If the list is statically known to be non-empty, either because its value is
known or because a proof already exists in the context, the proof can be
constructed automatically. 
Default implicit arguments allow this to happen silently. We define
\texttt{head} as follows:

\begin{SaveVerbatim}{defimp}

head : (xs : List a) -> 
       {default proof { trivial; } 
            p : isCons xs = True} -> a
head (x :: xs) = x

\end{SaveVerbatim}
\useverb{defimp} 

\noindent
The syntax \texttt{\{default val x : t\}} declares an implicit argument
\texttt{x} of type \texttt{t}, where if an explicit value is not given,
the value \texttt{val} is inserted.
Now when \texttt{head} is applied, the proof can be omitted. In the case that a
proof is not found, it can be provided explicitly as normal:

\begin{SaveVerbatim}{headapp}

head xs {p = ?headProof} 

\end{SaveVerbatim}
\useverb{headapp} 

[FIXME: It's not very satisfying that we use \texttt{tactics} here and
\texttt{proof} earlier, even though there is a technical reason to do so.]

\subsubsection{Conversions}

\Idris{} supports the creation of \emph{implicit conversions}, which allow
automatic conversion of values from one type to another when required to make
a term type correct. This is intended to increase convenience and reduce
verbosity in implemented embedded domain specific languages. 
For example, imagine a fragment of a typed DSL with string values and
a print operation:

\begin{SaveVerbatim}{exprstr}

data DSL    : Type -> Type where
     StrVal : String -> DSL String
     Print  : DSL String -> DSL ()
     ...

\end{SaveVerbatim}
\useverb{exprstr}

\noindent
When constructing a \texttt{Print} with a literal string, we must inject
a literal string into the \texttt{DSL} structure, as follows:

\begin{SaveVerbatim}{exprimpl1}

Print (StrVal "Hello") 

\end{SaveVerbatim}
\useverb{exprimpl1}

\noindent
When implementing \remph{embedded} DSLs, which we expect a programming to use
directly, this is unsatisfying --- it creates noise both for the reader and
the author of the program. To alleviate this problem, we have introduced
implicit conversions to \Idris{}, inspired by a similar feature in
Scala~\cite{Scala}:

\begin{SaveVerbatim}{exprstrimpl}

implicit MkStrVal : String -> DSL String
MkStrVal = StrVal

\end{SaveVerbatim}
\useverb{exprstrimpl}

\noindent
The effect of the \texttt{implicit} keyword before the \texttt{MkStrVal}
function is that the function will be applied to any expression of type
\texttt{String} where an expression of type \texttt{DSL String} is required
instead. This allows the above \texttt{Print} application to be written
as follows:

\useverb{exprstrimpl}

\begin{SaveVerbatim}{exprimpl2}

Print "Hello"

\end{SaveVerbatim}
\useverb{exprimpl2}

\noindent
Such conversions are, deliberately, limited. They cannot be chained, unlike
implicit coercions in Coq, to avoid coherence problems. Furthermore, to avoid
ambiguity problems, if there is more than one implicit conversion available
then \remph{neither} will be applied.





\section{Motivating Example: An Effectful Evaluator}

Let us begin by writing an evaluator for an evaluator for a simple expression
language, in \Idris{}. 
%We will introduce \Idris{} and its specific features in
%more depth in Section \ref{sect:idris}. For now, it suffices to know that
%\Idris{} syntax is based heavily on Haskell, with the exception that a 
%single colon is used for a type declaration and a double colon for list
%construction, emphasising the importance of types.
To keep the language as simple as possible, we will limit expressions
to integer values, and addition. In \Idris{}, we can represent expressions
as follows:

\begin{SaveVerbatim}{exprlang1}

data Expr = Val Int 
          | Add Expr Expr

\end{SaveVerbatim}
\useverb{exprlang1}

\noindent
Evaluating these expressions is straightforward, by traversing the expression
tree:

\begin{SaveVerbatim}{expreval1}

eval : Expr -> Int
eval (Val i) = i
eval (Add x y) = eval x + eval y

\end{SaveVerbatim}
\useverb{expreval1}

\noindent
However, for a realistic language we will need to add more features, such
as variables or user interaction, which require handling effects such as
exceptions (in the case of undefined variables) or state.

\subsection{An Effectful Evaluator}

\label{sect:naiveeval}

Let us briefly explore a na\"{i}ve implementation of an
evaluator for an expression language like the above, augmented with
variables, error checking and random numbers.

\subsubsection{Adding variables}

\begin{SaveVerbatim}{exprlang2}

data Expr = Val Int
          | Var String
          | Add Expr Expr

\end{SaveVerbatim}
\useverb{exprlang2}

To add variables, we introduce a state monad for threading an environment
through an evaluation.

\begin{SaveVerbatim}{evalstate}

Env : Type
Env = List (String, Int)

data Eval a = MkEval (Env -> (a, Env))

instance Monad Eval where
    ...

\end{SaveVerbatim}
\useverb{evalstate}

To access and update the environment, we need to provide primitives for
manipulating the state:

\begin{SaveVerbatim}{getputeval}

get : Eval Env
get = MkEval (\env => (env, env))
  
put : Env -> Eval ()
put env = MkEval (\x => ((), env))

\end{SaveVerbatim}
\useverb{getputeval}

\begin{SaveVerbatim}{expreval2}

eval : Expr -> Eval Int
eval (Val i) = return i
eval (Var x) = do env <- get
                  case lookup x env of
                       Just i => return i
eval (Add x y) = do x' <- eval x 
                    y' <- eval y
                    return (x' + y')

runEval : List (String, Int) -> Expr -> Int
runEval env expr = case eval expr of
                        MkEval f => do res <- f env
                                       return res

\end{SaveVerbatim}
\useverb{expreval2}

\noindent
\textbf{Aside:} \Idris{} supports idiom brackets~\cite{McBride2007} for a
more concise notation for programming with applicative functors (all monads
in \Idris{} are also applicative functions) which means the \texttt{Add}
case can be written as follows:

\begin{SaveVerbatim}{expridiom}

eval (Add x y) = [| eval x + eval y |]

\end{SaveVerbatim}
\useverb{expridiom}

\noindent
An application inside idiom brackets, \texttt{[| f a b c d |]} translates
directly to:

\begin{SaveVerbatim}{idiomtrans}

pure f <$> a <$> b <$> c <$> d

\end{SaveVerbatim}
\useverb{idiomtrans}

\subsubsection{Adding error checking}

\begin{SaveVerbatim}{evalmonad3}

data Eval a = MkEval (Env -> Maybe (a, Env))

lift : Maybe a -> Eval a

\end{SaveVerbatim}
\useverb{evalmonad3}

We need to modify \texttt{get} and \texttt{put} to inject a value into the
\texttt{Maybe} monad:

\begin{SaveVerbatim}{getput3}

get : Eval EvalState
get = MkEval (\env => Just (env, env))
  
put : EvalState -> Eval ()
put env = MkEval (\x => Just ((), env))

\end{SaveVerbatim}
\useverb{getput3}

\begin{SaveVerbatim}{expreval3}

eval : Expr -> Eval Int
eval (Val i) = return i
eval (Var x) = do env <- get
                  lift (lookup x env)
eval (Add x y) = [| eval x + eval y |]

\end{SaveVerbatim}
\useverb{expreval3}

\begin{SaveVerbatim}{runeval3}

runEval : List (String, Int) -> Expr -> Maybe Int
runEval env expr = case eval expr of
                        MkEval f => do (res, _) <- f env
                                       return res

\end{SaveVerbatim}
\useverb{runeval3}

\subsubsection{Adding random number generation}

\begin{SaveVerbatim}{exprlang4}

data Expr = Val Int
          | Var String
          | Add Expr Expr
          | Random Int

\end{SaveVerbatim}
\useverb{exprlang4}

We extend the evaluator state to include a random number seed as well as
the environment:

\begin{SaveVerbatim}{evalmonad4}

EvalState : Type
EvalState = (Int, List (String, Int))
  
data Eval a = MkEval (EvalState -> Maybe (a, EvalState))

\end{SaveVerbatim}
\useverb{evalmonad4}

Then, we can implement random number generation by extracting the appropriate
part of the state, calculating a number (here using a simple linear congruence
generator) the updating the seed:

\begin{SaveVerbatim}{rndint}

rndInt : Int -> Int -> Eval Int
rndInt lower upper 
    = do (seed, env) <- get
         let seed' = 1664525 * seed + 1013904223
         put (seed', env)
         return (seed `mod` (upper - lower) + lower)

\end{SaveVerbatim}
\useverb{rndint}

We need to modify the evaluator so that we extract the appropriate parts
of the state:

\begin{SaveVerbatim}{expreval4}

eval : Expr -> Eval Int
eval (Val i) = return i
eval (Var x) = do (seed, env) <- get
                  lift (lookup x env)
eval (Add x y) = [| eval x + eval y |]
eval (Random upper) = do val <- rndInt 0 upper
                         return val

\end{SaveVerbatim}
\useverb{expreval4}

Every time we add an effect, we have to update the monad and update any
code which manages the state. There are various ways we could improve this
--- for example implementing the state as a record and abstracting away
updates and queries --- but we have a more fundamental problem that the
effects we have added are neither \remph{composable} nor easily
\remph{reusable}, nor can we easily write subprograms which use a 
restricted subset of the effects.

\subsection{The Haskell approach: Monad transformers}

\begin{SaveVerbatim}{evalhs}

eval :: RandomGen g => 
        Expr -> RandT g (ReaderT Env Maybe) Int
eval (Val n) = return n
eval (Var x) = do env <- lift ask
                  lift (lift (lookup x env))
eval (Add x y) = pure (+) <*> eval x <*> eval y
eval (Random x) = do val <- getRandomR (0, x)
                     return val

\end{SaveVerbatim}
\useverb{evalhs}

Still problems: composability, noise of \texttt{lift} (though can be mitigated
by making type class versions, e.g. MonadReader class), still hard to call
a subset of effects, leads to tendency to return to an approach like
Section \ref{sect:naiveeval}, with minimal use of transformers.

\subsection{Can we do better?}



%\section{Algebraic Effects}




\section{\Eff{}: an Embedded DSL for Effects Management} 

In this section, I introduce \Eff{}, an embedded domain specific language
for managing computational effects in \Idris{}. I will introduce specific
distinctive features of \Idris{} as required --- in particular, we will
use implicit conversions and default implicit arguments in the implementation
of \Eff{} --- a full tutorial is available elsewhere~\cite{idristutorial}.  
First, I describe how to
use effects which are already defined in the language in order to
implement the evaluator described in the introduction. Then, I show how new
effects may be implemented. 

The framework consists of a DSL representation \texttt{Eff} for combining
effects, \texttt{EffM} for combining mutable effects, and implementations
of several predefined effects. We refer to the whole framework with the
name \Eff{}.

\label{sect:effdsl}

\subsection{Programming with \Eff{}}

Programs in the \Eff{} language are described using the following data type,
in the simplest case:

\begin{SaveVerbatim}{efftype}

Eff : (m  : Type -> Type) -> 
      (es : List EFFECT) -> (a  : Type) -> Type

\end{SaveVerbatim}
\useverb{efftype}

\noindent
Note that function types in \Idris{} take the form \texttt{(x : a) -> b}, with
an optional name \texttt{x} on the domain of the function. This is primarily
to allow the name \texttt{x} to be used in the codomain, although it is also
useful for documenting the purpose of an argument.

\texttt{Eff} is parameterised over a \remph{computation context}, \texttt{m}, which
describes the context in which the effectful program will be run, a list
of side effects \texttt{es} that the program is permitted to use, 
and the program's return type \texttt{a}. The name \texttt{m} for the computation
context is suggestive of a monad --- the computation context most commonly 
is a monad but there is no requirement for it to be so.

Side effects are described using the \texttt{EFFECT} type --- we will refer
to these as \remph{concrete} effects.  The following are 
among the predefined concrete effects:

\begin{SaveVerbatim}{effs}

STATE     : Type -> EFFECT
EXCEPTION : Type -> EFFECT
FILEIO    : Type -> EFFECT
STDIO     : EFFECT
RND       : EFFECT

\end{SaveVerbatim}
\useverb{effs}

\noindent
States are parameterised by the type of the state being carried, and exceptions
are parameterised by a type representing errors. File I/O is an effect which
allows a single file to be processed, with the type giving the current state
of the file (i.e. closed, open for reading, or open for writing). The
\texttt{STDIO} effect permits console I/O, and \texttt{RND} permits random
number generation.
%
For example, a program with some integer state, which performs console I/O 
and which could throw
an exception of type \texttt{Error} might have the following type:

\noindent
\begin{SaveVerbatim}{exprog}

 example : Eff IO [EXCEPTION Error, STDIO, STATE Int] ()

\end{SaveVerbatim}
\useverb{exprog}

\noindent
More generally, a program might modify the set of effects available. This
might be desirable for several reasons, such as adding a new effect, or to
update an index of a dependently typed state. In this case, we describe
programs using the \texttt{EffM} data type:

\begin{SaveVerbatim}{effmtype}

EffM : (m   : Type -> Type) -> 
       (es  : List EFFECT) -> 
       (es' : List EFFECT) -> 
       (a   : Type) -> Type

\end{SaveVerbatim}
\useverb{effmtype}

\noindent
\texttt{EffM} is parameterised over the context and type as before,
but separates input effects (\texttt{es}) from output effects (\texttt{es'}).
In fact, \texttt{Eff} is defined in terms of \texttt{EffM}, with equal
input/output effects.

We adopt the convention that the names \texttt{es} and \texttt{fs} refer to a list
in input effects, and the names \texttt{es'} and \texttt{fs'} refer to a list
of output effects.

\subsubsection{First example: State}

In general, an effectful program implemented in the \texttt{EffM} structure has
the look and feel of a monadic program in Haskell, since \texttt{EffM} supports
\texttt{do}-notation. To illustrate basic usage, let us implement
a program with state --- a function which tags each node in a binary tree with
a unique integer, counting depth first, left to right. We declare trees as
follows:

\begin{SaveVerbatim}{treedef}

data Tree a = Leaf 
            | Node (Tree a) a (Tree a)

\end{SaveVerbatim}
\useverb{treedef}

\noindent
To tag each node in the tree, we write an effectful program which, for each
node, tags the left subtree, reads and updates the state, tags the right
subtree, then returns a new node with its value tagged. The type
states that the program requires an integer state:

\begin{SaveVerbatim}{treelblty}

tag : Tree a -> Eff m [STATE Int] (Tree (Int, a))

\end{SaveVerbatim}
\useverb{treelblty}

\noindent
The implementation traverses the tree, using \texttt{get} and \texttt{put}
operations to read and write state:

\begin{SaveVerbatim}{treelbl}

tag Leaf = return Leaf
tag (Node l x r) 
     = do l' <- tag l
          lbl <- get
          put (lbl + 1)
          r' <- tag r
          return (Node l' (lbl, x) r')

\end{SaveVerbatim}
\useverb{treelbl}

\noindent
The \Eff{} system ensures, statically, that any
effectful functions which are called (\texttt{get} and \texttt{put} here)
require no more effects than are available.
The types of these functions are:

\begin{SaveVerbatim}{getputty}

get : Eff m [STATE x] x
put : x -> Eff m [STATE x] ()

\end{SaveVerbatim}
\useverb{getputty}

\noindent
Each effect is associated with a \remph{resource}. For example, the resource
associated with \texttt{STATE Int} is the integer state itself.  To \remph{run}
an effectful program, we must initialise each resource and instantiate
\texttt{m}. Here we instantiate \texttt{m} with \texttt{id}, resulting in a
pure function.

\begin{SaveVerbatim}{runEval}

tagFrom : Int -> Tree a -> Tree (Int, a)
tagFrom x t = runPure [x] (tag t)

\end{SaveVerbatim}
\useverb{runEval}

\noindent
In general, to run an effectful program, we use one of the functions
\texttt{run}, \texttt{runWith} or \texttt{runPure}, instantiating an
environment with resources corresponding to each effect:

\begin{SaveVerbatim}{runeffs}

run     : Applicative m => 
          Env m es -> EffM m es es' a -> m a
runWith : (a -> m a) -> 
          Env m es -> EffM m es es' a -> m a
runPure : Env id es -> EffM id es es' a -> a

\end{SaveVerbatim}
\useverb{runeffs}

\noindent
Corresponding functions \texttt{runEnv}, \texttt{runWithEnv} and
\texttt{runPureEnv} are also available for cases when the final resources are
required.  The only reason \texttt{run} needs \texttt{m} to be an applicative
functor
is that it uses \texttt{pure} to inject a pure value into \texttt{m}. If this
is inconvenient, \texttt{runWith} can be used instead. Note that, unlike the
monad transformer approach, there is no assumption or requirement that
\texttt{m} is a monad. Any type transformer is fine --- in particular,
if the effectful program can be translated into a pure function, \texttt{id}
is perfectly fine.

As we will see shortly, the particular choice of \texttt{m} can be
important. Consider, for example, the difference between running a program with
exceptions in the context of \texttt{IO}, \texttt{Maybe} or \texttt{Either}.
%
We will return to the definition of \texttt{Env} in Section \ref{sect:envdef}.
For the moment, it suffices to know that it is a heterogenous list of values
for the initial resources \texttt{es}.

\subsubsection{Labelled Effects}

When we invoke effectful functions such as \texttt{get} and \texttt{put},
the \Eff{} language internally searches through the list of available effects
to check that it is supported, and invokes the effect using the corresponding
resource. This leads to an important question: what if we have more than one 
effect which supports the function? A particular situation where this arises
is when we have more than one integer state.
%
For example, imagine we would like to count the number of \texttt{Leaf} nodes
in a tree while tagging nodes.  In this case, we will need two integer states:

\begin{SaveVerbatim}{treelblcount}

tagCount : Tree a -> 
     Eff m [STATE Int, STATE Int] (Tree (Int, a))

\end{SaveVerbatim}
\useverb{treelblcount}

\noindent
What should be the effect of \texttt{get} and \texttt{put} in this program?
Do they read and update the first state, the second, or choose
non-deterministically? 
%
In practice, the earlier effect is chosen. While clearly defined, this is
unlikely to be the desired behaviour, so 
to avoid this problem, effects may also be \remph{labelled} using the
\texttt{:::} operator.  A label can be of any type, and an
effect can be converted into a labelled effect using the \texttt{:-}
operator:

\begin{SaveVerbatim}{lbleff}

(:::) : lbl -> EFFECT -> EFFECT
(:-)  : (l : lbl) -> EffM m [x] [y] t -> 
                     EffM m [l ::: x] [l ::: y] t

\end{SaveVerbatim}
\useverb{lbleff}

\noindent
In order to implement \texttt{tagCount} now, first we define a type for the
labels. We have one state variable representing the leaf count, and one
representing the current tag:

\begin{SaveVerbatim}{lbltys}

data Vars = Count | Tag

\end{SaveVerbatim}
\useverb{lbltys}

\noindent
Then, we use these labels to disambiguate the states. To increment the count
at each leaf, we use \texttt{update}, which combines a \texttt{get} and a
\texttt{put} by applying a function to the state:

\begin{SaveVerbatim}{treelblcountty}

tagCount : Tree a -> Eff m [Tag   ::: STATE Int, 
                            Count ::: STATE Int] 
                              (Tree (Int, a))
\end{SaveVerbatim}
\begin{SaveVerbatim}{treelblcountdef}
tagCount Leaf
     = do Count :- update (+1)
          return Leaf
tagCount (Node l x r) 
     = do l' <- tagCount l
          lbl <- Tag :- get
          Tag :- put (lbl + 1)
          r' <- tagCount r
          return (Node l' (lbl, x) r')

\end{SaveVerbatim}

\useverb{treelblcountty}

\useverb{treelblcountdef}

\noindent
In order to get the count, we will need access to the environment \remph{after}
running \texttt{tagCount}. To do so, we use \texttt{runPureEnv}, which returns
the final resource states as well as the result of the computation:

\begin{SaveVerbatim}{runenvty}

runPureEnv : Env id xs -> 
             EffM id xs xs' a -> (Env id xs', a)

\end{SaveVerbatim}
\useverb{runenvty}

\noindent
To initialise the environment, we give the label name along with the initial
value of the resource:

\begin{SaveVerbatim}{rpenvtree}

runPureEnv [Tag := 0, Count := 0] (tagCount t) 

\end{SaveVerbatim}
\useverb{rpenvtree}

\noindent
And finally, to implement a pure wrapper function which returns a pair of the
count of leaves and a labelled tree, we call \texttt{runPureEnv} with the
initial resources, and match on the returned resources to retrieve the leaf
count:

\begin{SaveVerbatim}{lcfrom}

tagCountFrom : Int -> Tree a -> (Int, Tree (Int, a))
tagCountFrom x t 
    = let ([_, Count := leaves], tree) =
       runPureEnv [Tag := 0, Count := 0] (tagCount t)
          in (leaves, tree)

\end{SaveVerbatim}
\useverb{lcfrom}

\subsubsection{An Effectful Evaluator revisited}

Recall the effectful evaluator in the introduction. To implement this
in the \Eff{} language, we must support exceptions, a state containing the
current environment, and random number generation. Environments are
represented as a list of mappings from \texttt{String} to \texttt{Int}:

\begin{SaveVerbatim}{langenv}

Vars : Type
Vars = List (String, Int)

\end{SaveVerbatim}
\useverb{langenv}

\noindent
The evaluator invokes supported effects where needed. We use the following
effectful functions:

\begin{SaveVerbatim}{efftypes}

get    : Eff m [STATE x] x
raise  : a -> Eff m [EXCEPTION a] b
rndInt : Int -> Int -> Eff m [RND] Int

\end{SaveVerbatim}
\useverb{efftypes}

\noindent
The evaluator itself is written in the \texttt{Eff} type.
%, using
%\texttt{do}-notation as in the monad transformer implementation. 
The 
type, again, lists the resources we require:

\begin{SaveVerbatim}{langeffty}

eval : Expr -> 
       Eff m [EXCEPTION String, RND, STATE Vars] t

\end{SaveVerbatim}
\useverb{langeffty}

\noindent
The implementation simply invokes the required effects \texttt{get},
\texttt{raise} and \texttt{rndInt}, with the \Eff{} system checking that
these effects are available:

\begin{SaveVerbatim}{langeff}

eval (Val x) = return x
eval (Var x) = 
    do vs <- get
       case lookup x vs of
            Nothing => raise ("Error " ++ x)
            Just val => return val
eval (Add l r) = [| eval l + eval r |]
eval (Random upper) = rndInt 0 upper

\end{SaveVerbatim}
\useverb{langeff}

\noindent
\textbf{Remark:}
We have used idiom brackets~\cite{McBride2007} in this implementation, to
give a more concise notation for applicative programming with effects.
An application inside idiom brackets, \texttt{[| f a b c d |]} translates
directly to:

\begin{SaveVerbatim}{idiomtrans}

pure f <$> a <$> b <$> c <$> d

\end{SaveVerbatim}
\useverb{idiomtrans}

\noindent
In order to run this evaluator, we must provide initial values for the resources
associated with each effect. Exceptions require the unit resource, random
number generation requires an initial seed, and the state requries an initial
environment. We instantiate \texttt{m} with \texttt{Maybe} to be able
to handle exceptions:

\begin{SaveVerbatim}{exprrun}

runEval : List (String, Int) -> Expr -> Maybe Int
runEval env expr = run [(), 123456, env] (eval expr)

\end{SaveVerbatim}
\useverb{exprrun}

\noindent
Extending the evaluator with a new effect, such as \texttt{STDIO} is a matter
of extending the list of available effects in its type.  We could use this, for
example, to print out the generated random numbers:

\begin{SaveVerbatim}{langeffio}

eval : Expr -> 
       Eff m [EXCEPTION String, STDIO, 
              RND, STATE Vars] t
...
eval (Random upper) = do num <- rndInt 0 upper
                         putStrLn (show num)
                         return num

\end{SaveVerbatim}
\useverb{langeffio}

\noindent
We can insert the \texttt{STDIO} effect anywhere in the list without difficulty
--- the only requirements are that its initial resource is in the corresponding
position in the call to \texttt{run}, and that \texttt{run} instantiates
a context which supports \texttt{STDIO}, such as \texttt{IO}:

\begin{SaveVerbatim}{exprrun}

runEval : List (String, Int) -> Expr -> IO Int
runEval env expr 
    = run [(), (), 123456, env] (eval expr)

\end{SaveVerbatim}
\useverb{exprrun}
\subsection{Implementing effects}

In order to implement a new effect, we define a new type (of kind \texttt{Effect})
and explain how that effect is interpreted in some underlying context
\texttt{m}. An \texttt{Effect} describes an effectful computation,
parameterised by an input resource \texttt{res}, an output resource \texttt{res'}, 
and the type of the computation \texttt{t}.

\begin{SaveVerbatim}{effectty}

Effect : Type
Effect = (res : Type) -> (res' : Type) -> 
         (t : Type) -> Type

\end{SaveVerbatim}
\useverb{effectty}

\noindent
Effects are typically described as algebraic data types. To \remph{run} an
effect, they must be handled in some specific computation context \texttt{m}.
We achieve this by making effects and contexts instances of a a type class,
\texttt{Handler}, which has a method \texttt{handle} explaining this
interpretation:

\begin{SaveVerbatim}{effh}

class Handler (e : Effect) (m : Type -> Type) where
     handle : res -> (eff : e res res' t) -> 
              (res' -> t -> m a) -> m a

\end{SaveVerbatim}
\useverb{effh}

\noindent
Type classes in \Idris{} may be parameterised by anything --- not only types,
but also values, and even other type classes. Thus, if a parameter is anything
other than a \texttt{Type}, it must be given a type label explicitly, like
\texttt{e} and \texttt{m} here.

Handlers are parameterised by the effect they handle, and the context in which
they handle the effect. This allows several different context-dependent
handlers to be written --- e.g. exceptions could be handled differently in an
\texttt{IO} setting than in a \texttt{Maybe} setting. When effects are combined,
as in the evaluator example, all effects must be handled
in the context in which the program is run.

An effect \texttt{e res res' t} updates a resource type \texttt{res} to a
resource type \texttt{res'}, returning a value \texttt{t}. The handler, therefore,
implements this update in a context \texttt{m} which may support side effects.
The handler is written in continuation passing style. This is for two reasons:
Firstly, it returns two values, a new resource and the result of the computation,
which is more cleanly managed in a continuation than by returning a tuple;
secondly, and more importantly, it gives the handler the flexibility to invoke
the continuation any number of times (zero or more).

An \texttt{Effect}, which is the internal algebraic description of an effect,
is promoted into a concrete \texttt{EFFECT}, which is expected by the
\texttt{EffM} structure, with the \texttt{MkEff} constructor:

\begin{SaveVerbatim}{efft}

data EFFECT : Type where
     MkEff : Type -> Effect -> EFFECT

\end{SaveVerbatim}
\useverb{efft}

\noindent
\texttt{MkEff} additionally records the resource state of an effect.
In the remainder of this section, we describe how several effects can be
implemented in this way: mutable state; console I/O; exceptions; files; random
numbers, and non-determinism.

\subsubsection{State}

In general, effects are described algebraically in terms of the operations
they support. In the case of \texttt{State}, the supported effects are reading
the state (\texttt{Get}) and writing the state (\texttt{Put}).

\begin{SaveVerbatim}{statealg}

data State : Effect where
     Get :      State a a a
     Put : b -> State a b ()

\end{SaveVerbatim}
\useverb{statealg}

\noindent
The resource associated with a state corresponds to the state itself. So,
the \texttt{Get} operation leaves this state intact (with a resource type
\texttt{a} on entry and exit) but the \texttt{Put} operation may update this
state (with a resource type \texttt{a} on entry and \texttt{b} on exit)
--- that is, a \texttt{Put} may update the type of the stored value.
We can implement a handler for this effect, for all contexts \texttt{m},
as follows:

\begin{SaveVerbatim}{statehandle}

instance Handler State m where
     handle st Get     k = k st st
     handle st (Put n) k = k n ()

\end{SaveVerbatim}
\useverb{statehandle}

\noindent
When running \texttt{Get}, the handler passes the current state to the
continuation as both the new resource value (the first argument of the
continuation \texttt{k}) as well as the return value of the computation (the
second argument of the continuation). When running \texttt{Put}, the new state
is passed to the continuation as the new resource value.

We then convert the algebraic effect \texttt{State} to a concrete
effect usable in an \Eff{} program using the \texttt{STATE} function, to which
we provide the initial state type as folllows:

\begin{SaveVerbatim}{statepromote}

STATE : Type -> EFFECT
STATE t = MkEff t State

\end{SaveVerbatim}
\useverb{statepromote}

\noindent
We adopt the convention that algebraic effects, of type \texttt{Effect},
have an initial upper case letter. Concrete effects, of type \texttt{EFFECT},
are correspondingly in all upper case.

Algebraic effects are promoted to \Eff{} programs with concrete effects
by using the
\texttt{mkEffect} function. We will postpone giving the type of
\texttt{mkEffect} until Section \ref{sect:mkeffect} --- for now,
it suffices to know that it converts an
\texttt{Effect} to an effectful program. To create the \texttt{get} and
\texttt{put} functions, for example:

\begin{SaveVerbatim}{getdef}

get : Eff m [STATE x] x
get = mkEffect Get 

\end{SaveVerbatim}
\useverb{getdef}

\begin{SaveVerbatim}{putdef}
put : x -> Eff m [STATE x] ()
put val = mkEffect (Put val)

\end{SaveVerbatim}
\useverb{putdef}

\noindent
We may also find it useful to mutate the \remph{type} of a state, considering
that states may themselves have dependent types (we may, for example, add
an element to a vector in a state). The \texttt{Put} constructor supports this,
so we can implement \texttt{putM} to update the state's type:

\begin{SaveVerbatim}{putmdef}

putM : y -> EffM m [STATE x] [STATE y] ()
putM val = mkEffect (Put val)

\end{SaveVerbatim}
\useverb{putmdef}

\noindent
Finally, it may be useful to combine \texttt{get} and \texttt{put} in a single
update:

\begin{SaveVerbatim}{update}

update : (x -> x) -> Eff m [STATE x] ()
update f = do val <- get; put (f val) 

\end{SaveVerbatim}
\useverb{update}

\subsubsection{Console I/O}

We consider a simplified version of console I/O which supports reading
and writing strings to and from the console. There is no associated resource,
although in an alternative implementation we may associate it with an abstract
world state, or a pair of file handles for \texttt{stdin}/\texttt{stdout}.
Algebraically:
%we describe console I/O as follows:

\begin{SaveVerbatim}{stdioeff}

data StdIO : Effect where
     PutStr : String -> StdIO () () ()
     GetStr : StdIO () () String

STDIO : EFFECT
STDIO = MkEff () StdIO

\end{SaveVerbatim}
\useverb{stdioeff}

\noindent
The obvious way to handle \texttt{StdIO} is by translating via the \texttt{IO}
monad, which is implemented straightforwardly as follows:

\begin{SaveVerbatim}{stdiohandle}

instance Handler StdIO IO where
    handle () (PutStr s) k = do putStr s; k () ()
    handle () GetStr     k = do x <- getLine; k () x 

\end{SaveVerbatim}
\useverb{stdiohandle}

\noindent
Unlike the \texttt{State} effect, for which the handler worked in \remph{all}
contexts, this handler only applies to effectful programs run in an \texttt{IO}
context. We can implement alternative handlers, and indeed there is no
reason that effectful programs in \texttt{StdIO} must be evaluated in a monadic
context. For example, we can define I/O stream functions:
%and an associated handler:

\begin{SaveVerbatim}{iostream}

data IOStream a 
   = MkStream (List String -> (a, List String))

instance Handler StdIO IOStream where
    ...
\end{SaveVerbatim}
\useverb{iostream}

\noindent
A handler for \texttt{StdIO} in \texttt{IOStream} context generates a function
from a list of strings (the input text) to a value and the output text. We
can build a pure function which simulates real console I/O:

\begin{SaveVerbatim}{mkstrfunty}

mkStrFn : Env IOStream xs -> Eff IOStream xs a -> 
          List String -> (a, List String)
\end{SaveVerbatim}
\useverb{mkstrfunty}

\begin{SaveVerbatim}{mkstrfun}
mkStrFn {a} env p input = case mkStrFn' of
                               MkStream f => f input
  where injStream : a -> IOStream a
        injStream v = MkStream (\x => (v, []))
        mkStrFn' : IOStream a
        mkStrFn' = runWith injStream env p

\end{SaveVerbatim}
\useverb{mkstrfun}

\noindent
To illustrate this, we write a simple console I/O program which runs in
any context which has a handler for \texttt{StdIO}:

\begin{SaveVerbatim}{ioname}

name : Handler StdIO e => Eff e [STDIO] ()
name = do putStr "Name? "
          n <- getStr
          putStrLn ("Hello " ++ show n)

\end{SaveVerbatim}
\useverb{ioname}

\noindent
Using \texttt{mkStrFn}, we can run this as a pure function which uses a list
of strings as its input, and gives a list of strings as its output. We can
evaluate this at the \Idris{} prompt:

\begin{SaveVerbatim}{mkstrfunrun}

*name> show $ mkStrFn [()] name ["Edwin"]
((), ["Name?" , "Hello Edwin\n"]) 

\end{SaveVerbatim}
\useverb{mkstrfunrun}

\noindent

\begin{SaveVerbatim}{blehlatex}
$
\end{SaveVerbatim}

\noindent
This suggests that alternative, pure, handlers for console I/O,
or any I/O effect, can be used for unit testing and reasoning about I/O programs
without executing any real I/O.
%and possibly even proving theorems about them.

\subsubsection{Exceptions}

The exception effect supports only one operation, \texttt{Raise}.
Exceptions are parameterised over an error type \texttt{e}, so \texttt{Raise}
takes a single argument to represent the error. The associated resource is
of unit type, and since raising an exception causes computation to abort, 
raising an exception can return a value of any type.
%, since it will not be used.

\begin{SaveVerbatim}{exctype}

data Exception : Type -> Effect where
     Raise : e -> Exception e () () b 

EXCEPTION : Type -> EFFECT
EXCEPTION e = MkEff () (Exception e) 

\end{SaveVerbatim}
\useverb{exctype}

\noindent
The semantics of \texttt{Raise} is to abort computation, therefore handlers
of exception effects do not call the continuation \texttt{k}. In any case, 
this should be impossible since passing the result to the continuation would
require the ability to invent a value in any arbitrary type \texttt{b}!
The simplest handler runs in a \texttt{Maybe} context:

\begin{SaveVerbatim}{excmaybe}

instance Handler (Exception a) Maybe where
     handle _ (Raise err) k = Nothing

\end{SaveVerbatim}
\useverb{excmaybe}

\noindent
Exceptions can be handled in any context which supports some representation of
failed computations. In an \texttt{Either e} context, for example, we can
use \texttt{Left} to represent the error case:

\begin{SaveVerbatim}{exceither}

instance Handler (Exception e) (Either e) where
     handle _ (Raise err) k = Left err

\end{SaveVerbatim}
\useverb{exceither}

\noindent
Given that we can raise exceptions in an \Eff{} program, it is also useful to be
able to catch them. The \texttt{catch} operation runs a possibly failing
computation \texttt{comp} in some context \texttt{m}, running \texttt{handler}
if the computation fails:

\begin{SaveVerbatim}{catch}

catch : Catchable m err =>
        (comp : EffM m xs xs' a) -> 
        (handler : err -> EffM m xs xs' a) ->
        EffM m xs xs' a

\end{SaveVerbatim}
\useverb{catch}

\noindent
Using \texttt{catch} requires that the computation context \texttt{m} supports
throwing and catching of errors, using the \texttt{Catchable} class implemented
as part of the \Idris{} library:

\begin{SaveVerbatim}{catchable}

class Catchable (m : Type -> Type) t where
    throw : t -> m a
    catch : m a -> (t -> m a) -> m a

\end{SaveVerbatim}
\useverb{catchable}

\noindent
This is implemented directly as part of the \Eff{} DSL.
%--- any effectful
%computation may fail, and therefore implement a handler in a \texttt{Catchable}
%context after all. 
We have a simple instance for \texttt{Maybe}, provided that
the error thrown is the unit error:

\begin{SaveVerbatim}{catchablemaybe}

instance Catchable Maybe () where
    throw () = Nothing
    catch Nothing  h = h ()
    catch (Just x) h = Just x

\end{SaveVerbatim}
\useverb{catchablemaybe}

\noindent
The instance for \texttt{Either e} is similar, except that any type may be
thrown as an error:

\begin{SaveVerbatim}{catchableeither}

instance Catchable (Either e) e where
    throw x = Left x
    catch (Left err) h = h err
    catch (Right x)  h = (Right x)

\end{SaveVerbatim}
\useverb{catchableeither}

\subsubsection{Random numbers}

Random number generation can be implemented as an effect, with the resource
tracking the \remph{seed} from which the next number will be generated.
The \texttt{Random} effect supports one operation, \texttt{getRandom}, which
requires an \texttt{Int} resource and returns the next number:

\begin{SaveVerbatim}{randomeff}

data Random : Type -> Type -> Type -> Type where
     GetRandom : Random Int Int Int

RND : EFFECT
RND = MkEff Int Random

\end{SaveVerbatim}
\useverb{randomeff}

\noindent
Handling random number generation shows that it is a state effect in
disguise, where the effect updates the seed.
This is a simple linear congruential pseudo-random number generator:

\begin{SaveVerbatim}{randomhandle}

instance Handler Random m where
    handle seed GetRandom k
         = let seed' = 1664525 * seed + 1013904223 in
               k seed' seed'

\end{SaveVerbatim}
\useverb{randomhandle}

\noindent
Alternative handlers could use a different, possibly more secure approach.
In any case, we can implement a top level function which returns a random
number between a lower and upper bound as follows:

\begin{SaveVerbatim}{rndint}

rndInt : Int -> Int -> Eff m [RND] Int
rndInt lower upper 
    = do v <- mkEffect GetRandom 
         return (v `mod` (upper - lower) + lower)

\end{SaveVerbatim}
\useverb{rndint}

\subsubsection{Resource management: Files}

Instead of implementing all I/O operations as a single effect, as with the
\texttt{IO} monad, we can separate operations into more fine-grained effects.
After Console I/O, another I/O related effect which we can handle separately
is file management. Here, we can take advantage of the \remph{resource}
associated with an effect, and the fact that resource types are \remph{mutable},
by associating the file handling effect with an individual file, parameterised
by its current state (i.e. closed, open for reading, or open for writing).
File handles are represented as follows, where \texttt{File} is a primitive
file handle provided by the \Idris{} library:

\begin{SaveVerbatim}{fileeff}

data Mode = Read | Write

data OpenFile : Mode -> Type where
     FH : File -> OpenFile m

\end{SaveVerbatim}
\useverb{fileeff}

\noindent
When we declare the \texttt{FileIO} algebraic effect type, we express in
the resource transitions how each effect changes the state of the resource:

\begin{SaveVerbatim}{fileio}

data FileIO : Effect where
     Open  : String -> (m : Mode) -> 
             FileIO () (OpenFile m) ()
     Close : FileIO (OpenFile m) () ()
\end{SaveVerbatim}
\useverb{fileio}

\begin{SaveVerbatim}{fileio2}
     ReadLine  : FileIO (OpenFile Read)  
                        (OpenFile Read) String
     WriteLine : String -> 
                 FileIO (OpenFile Write) 
                        (OpenFile Write) ()
     EOF       : FileIO (OpenFile Read)  
                        (OpenFile Read) Bool

\end{SaveVerbatim}
\useverb{fileio2}

\noindent
We can see from this declaration that opening a file moves from an empty
resource to a file open for a specific purpose and that closing a file removes
the file resource. Only files which are open for reading may be read, or tested
for end of file, and only files open for writing may be written to. Any
violation of this resource access protocol will result in a type error.
In general, we can use the effect signature of a function to manage resources
safely, subsuming the resource management DSL we have previously
implemented~\cite{bradyresource}.

The \texttt{FILE\_IO} effect is parameterised over the current state of a
file resource with which it is associated:

\begin{SaveVerbatim}{fileres}

FILE_IO : Type -> EFFECT

\end{SaveVerbatim}
\useverb{fileres}

\noindent
The type of \texttt{open} expresses that the resource changes from a unit
to an open file:

\begin{SaveVerbatim}{openeff}

open : String -> (m : Mode) -> 
       EffM IO [FILE_IO ()] [FILE_IO (OpenFile m)] ()

\end{SaveVerbatim}
\useverb{openeff}

\noindent
Note that opening a file may fail --- we will deal with exceptional behaviour
shortly. 
Using \texttt{EffM}, we have expressed that opening a file causes a change in the
resource state.
It is then only possible to close a file if there is an open file
available:

\begin{SaveVerbatim}{closeeff}

close : EffM IO [FILE_IO (OpenFile m)] [FILE_IO ()] ()

\end{SaveVerbatim}
\useverb{closeeff}

\noindent
Reading is only possible from a file opened for reading:

\begin{SaveVerbatim}{readeff}

readLine : Eff IO [FILE_IO (OpenFile Read)] String

\end{SaveVerbatim}
\useverb{readeff}

\noindent
As with \texttt{STATE}, we can use labelled resources if we require more than
one file handle at a time. We have handlers for \texttt{FileIO} for the
\texttt{IO} context, which does not handle exceptions (in which case failing
to open a file is a fatal run-time error), and an \texttt{IOExcept e}
context which is \texttt{IO} augmented with exceptions of type \texttt{e}
and an instance of the \texttt{Catchable} class:

\begin{SaveVerbatim}{ioexcept}

data IOExcept err a
ioe_lift : IO a -> IOExcept err a

instance Catchable IOExcept err

\end{SaveVerbatim}
\useverb{ioexcept}

\noindent
Assuming we are in a state where we have a file handle available and open
for reading, we can read the contents of a file into a list of strings:

\noindent
\begin{SaveVerbatim}{readfile}

 readLines : Eff (IOExcept String) 
                [FILE_IO (OpenFile Read)] (List String)
 readLines = readAcc [] where
   readAcc : List String -> 
             Eff (IOExcept String) 
                 [FILE_IO (OpenFile Read)] (List String)
   readAcc acc = do e <- eof
                    if (not e)
                       then do str <- readLine
                               readAcc (str :: acc)
                       else return (reverse acc)

\end{SaveVerbatim}
\useverb{readfile}

\noindent
To read a file, given a file path, into a list of strings, reporting an
error where necessary, we can write the following \texttt{readFile}
function. We add \texttt{STDIO} to the list of effects so that we can report
an error to the console if opening the file fails:

\noindent
\begin{SaveVerbatim}{readfileproto}

 readFile : String -> Eff (IOExcept String)
                [FILE_IO (), STDIO] (List String)
 readFile path = catch (do open path Read
                           lines <- readLines
                           close
                           return lines)
         (\err => do putStrLn ("Failed " ++ err)
                     return [])

\end{SaveVerbatim}
\useverb{readfileproto}

\noindent
The effect type of \texttt{readFile} means that we must begin \remph{and} end
with no open file. This means that omitting the \texttt{close} would result in
a compile time type error. It would also be a type error to try to invoke
\texttt{readLines} before the file was open, or if the file was opened for
writing instead.

\subsubsection{Non-determinism}

Following~\cite{Bauer},
non-determinism can be implemented as an effect \texttt{Selection},
in which a \texttt{Select} operation chooses one value non-deterministically
from a list of possible values:

\begin{SaveVerbatim}{selecteff}

data Selection : Effect where
     Select : List a -> Selection () () a

\end{SaveVerbatim}
\useverb{selecteff}

\noindent
We can handle this effect in a \texttt{Maybe} context, trying every choice
in a list given to \texttt{Select} until the computation succeeds:

\begin{SaveVerbatim}{selectmaybe}

instance Handler Selection Maybe where
     handle _ (Select xs) k = tryAll xs where
         tryAll [] = Nothing
         tryAll (x :: xs) = case k () x of
                                 Nothing => tryAll xs
                                 Just v => Just v

\end{SaveVerbatim}
\useverb{selectmaybe}

\noindent
The handler for \texttt{Maybe} produces one result if it exists, effectively
performing a depth first search of all of the values passed to \texttt{Select}.
Note in particular that the handler runs the continuation for every element
of the list until the result of running the continuation succeeds.

Alternatively, we can find \remph{every} possible result by handling selection
in a \texttt{List} context:

\begin{SaveVerbatim}{selectlist}

instance Handler Selection List where
     handle r (Select xs) k = concatMap (k r) xs

\end{SaveVerbatim}
\useverb{selectlist}

\noindent
We can use the \texttt{Selection} effect to implement search problems by
non-deterministically choosing from a list of candidate solutions. For example,
a solution to the n-Queens problem can be implemented as follows.
First, we write a function which checks whether a point on a chessboard
attacks another if occupied by a Queen:

\begin{SaveVerbatim}{noattack}

no_attack : (Int, Int) -> (Int, Int) -> Bool
no_attack (x, y) (x', y')
  = x /= x' && y /= y' && abs (x - x') /= abs (y - y')

\end{SaveVerbatim}
\useverb{noattack}

\noindent
Then, given a column and a list of Queen positions, we find the rows on which
a Queen may safely be placed in that column using a list comprehension:

\begin{SaveVerbatim}{rowsin}

rowsIn : Int -> List (Int, Int) -> List Int
rowsIn col qs 
   = [ x | x <- [1..8], all (no_attack (x, col)) qs ]

\end{SaveVerbatim}
\useverb{rowsin}

\noindent
Finally, we compute a solution by accumulating a set of Queen positions,
column by column, non-deterministically choosing a position for a Queen in
each column.

\begin{SaveVerbatim}{nqueens}

addQueens : Int -> List (Int, Int) -> 
            Eff m [SELECT] (List (Int, Int))
addQueens 0   qs = return qs
addQueens col qs 
   = do row <- select (rowsIn col qs)
        addQueens (col - 1) ((row, col) :: qs)

\end{SaveVerbatim}
\useverb{nqueens}

\noindent
We can run this in \texttt{Maybe} context, to retrieve one solution, or in
\texttt{List} context, to retrieve all solutions. In a
\texttt{Maybe} context, for example, we can define:

\begin{SaveVerbatim}{getqueens}

getQueens : Maybe (List (Int, Int))
getQueens = run [()] (addQueens 8 [])

\end{SaveVerbatim}
\useverb{getqueens}

\noindent
Then to find the first solution, we run \texttt{getQueens} at the %\Idris{}
REPL:

\begin{SaveVerbatim}{queenmaybe}

*Queens> show getQueens
"Just [(4, 1), (2, 2), (7, 3), (3, 4), 
       (6, 5), (8, 6), (5, 7), (1, 8)]" : String

\end{SaveVerbatim}
\useverb{queenmaybe}

\noindent
\textbf{Remark:} It is important to note that when combining \texttt{SELECT}
with other effects, the values of other resources are reset at the beginning
of each \texttt{select} branch. This means, at least in the current implementation,
that state cannot be shared between branches. While perhaps not so important
for selection, this may be a problem for other control effects such as 
co-operative multithreading, for which we may need a more flexible handler if
we wish to deal with shared state. We will deal with this issue in future work.

\subsection{Effect Polymorphism}

Since \Eff{} is implemented as an embedding in a host language, we can exploit
features of the host language. The means that we can write higher order
functions, and functions which are polymorphic in their effects. For example,
a \texttt{mapEff} function can be implemented corresponding to \texttt{fmap}
for functors:

\noindent
\begin{SaveVerbatim}{mapeff}

mapEff : Applicative m => 
         (a -> Eff m xs b) -> List a -> Eff m xs (List b)
mapEff f []        = pure [] 
mapEff f (x :: xs) = [| f x :: mapEff f xs |]

\end{SaveVerbatim}
\useverb{mapeff}

\noindent
This applies an effectful function across a list, provided that we are in
an applicative context, and that the effectful function supports the correct
set of effects.


\section{The \Eff{} DSL implementation}

\begin{SaveVerbatim}{effdslstruct}
data EffM : (m : Type -> Type) -> List EFFECT -> List EFFECT -> Type -> Type where
     value   : a -> EffM m xs xs a
     (>>=)   : EffM m xs xs' a -> (a -> EffM m xs' xs'' b) -> EffM m xs xs'' b
     effectP : (prf : EffElem e a xs) -> (eff : e a b t) -> EffM m xs (updateResTy xs prf eff) t
     liftP   : (prf : SubList ys xs) -> EffM m ys ys' t -> EffM m xs (updateWith ys' xs prf) t
     (:-)    : (l : ty) -> EffM m [x] [y] t -> EffM m [l ::: x] [l ::: y] t
     new     : Handler e m => res -> EffM m (MkEff res e :: xs) (MkEff res' e :: xs') a -> EffM m xs xs' a
     catch   : Catchable m err => EffM m xs xs' a -> (err -> EffM m xs xs' a) -> EffM m xs xs' a
\end{SaveVerbatim}

\begin{figure*}[t]
\begin{center}
\useverb{effdslstruct}
\end{center}
\caption{The \Eff{} DSL data type}
\label{effdsltype}
\end{figure*}

\subsection{Language representation}

The \Eff{} langued is implemented as a syntax tree \texttt{EffM}
plus a tagless interpreter, where the
types in the syntax tree ensure that all \Eff{} programs are well typed.
The language itself as parameterised over its computation context \texttt{m},
and indexed by the list of effects on input and the list of effects on output,
as well as the return type of the computation:

\begin{SaveVerbatim}{effty}

EffM : (m : Type -> Type) -> 
       List EFFECT -> List EFFECT -> Type -> Type

\end{SaveVerbatim}
\useverb{effty}

\noindent
For the common case of programs in which the input effects are the same as
the output effects, we define \texttt{Eff}:

\begin{SaveVerbatim}{effimmutable}

Eff : (Type -> Type) -> List EFFECT -> Type -> Type
Eff m xs t = EffM m xs xs t

\end{SaveVerbatim}
\useverb{effimmutable}

\noindent
The complete syntax is given in Figure \ref{effdsltype} for reference. In this
section, we describe the constructs in detail.

\subsubsection{Basic constructs}

In the simplest case, we would like to inject pure values into the
\texttt{EffM} representation:

\begin{SaveVerbatim}{effval}

value : a -> EffM m xs xs a

\end{SaveVerbatim}
\useverb{effval}

We have \texttt{(>==)} to support \texttt{do} notation:

\begin{SaveVerbatim}{effbind}

(>>=) : EffM m xs xs' a -> 
        (a -> EffM m xs' xs'' b) -> EffM m xs xs'' b

\end{SaveVerbatim}
\useverb{effbind}

\subsubsection{Invoking effects}

\subsubsection*{Directly invoking effects}

\begin{SaveVerbatim}{effinvoke}

effectP : (prf : EffElem e a xs) -> (eff : e a b t) -> 
          EffM m xs (updateResTy xs prf eff) t

\end{SaveVerbatim}
\useverb{effinvoke}

\begin{SaveVerbatim}{}

updateResTy : (xs : List EFFECT) -> 
              EffElem e a xs -> e a b t -> List EFFECT
updateResTy {b} (MkEff a e :: xs) Here      n 
                 = (MkEff b e) :: xs
updateResTy     (x :: xs)         (There p) n 
                 = x :: updateResTy xs p n

\end{SaveVerbatim}
\useverb{}

\begin{SaveVerbatim}{effelem}

data EffElem : (Type -> Type -> Type -> Type) -> 
               Type -> List EFFECT -> Type where
     Here : EffElem x a (MkEff a x :: xs)
     There : EffElem x a xs -> EffElem x a (y :: xs)

\end{SaveVerbatim}
\useverb{effelem}

\Idris{} currently has very limited proof search capabilities, but they
are sufficient for constructing proofs of \texttt{EffElem x xs} automatically.
If a list is statically known, we can find a proof
by a brute force search, if \texttt{x} is an available effect. If not, we limit
the search depth:

\begin{SaveVerbatim}{findeff}

findEffElem : Nat -> Tactic 
findEffElem O = Refine "Here" `Seq` Solve 
findEffElem (S n) = GoalType "EffElem" 
      (Try (Refine "Here" `Seq` Solve)
           (Refine "There" `Seq` 
                   (Solve `Seq` findEffElem n)))

\end{SaveVerbatim}
\useverb{findeff}

\noindent
This could be improved with a simple reflection mechanism to allow the proof
search to inspect the goal --- this will be added in a future version of \Idris{}.
In practice, we have found the brute force approach to be adequate. Once we
have proof search, we can use the default implicit argument mechanism to
invoke proof search automatically:

\begin{SaveVerbatim}{impeff}

effect : {default tactics { 
                     reflect findEffElem 100; 
                     solve; 
                  } 
            prf : EffElem e a xs} -> 
         (eff : e a b t) -> 
         EffM m xs (updateResTy xs prf eff) t
effect {prf} e = effectP prf e

\end{SaveVerbatim}
\useverb{impeff}

\subsubsection*{Effectful subprograms}

\begin{SaveVerbatim}{efflift}

lift : (prf : SubList ys xs) ->
       EffM m ys ys' t -> 
       EffM m xs (updateWith ys' xs prf) t

\end{SaveVerbatim}
\useverb{efflift}

\begin{SaveVerbatim}{sublist}

data SubList : List a -> List a -> Type where
     SubNil : SubList [] []
     Keep   : SubList xs ys -> 
              SubList (x :: xs) (x :: ys)
     Drop   : SubList xs ys -> 
              SubList xs (x :: ys)

\end{SaveVerbatim}
\useverb{sublist}

\begin{SaveVerbatim}{updwith}

updateWith : (ys' : List a) -> (xs : List a) ->
             SubList ys xs -> List a
updateWith (y :: ys) (x :: xs) (Keep rest) 
           = y :: updateWith ys xs rest
updateWith ys        (x :: xs) (Drop rest) 
           = x :: updateWith ys xs rest
updateWith []        []        SubNil      
           = []

\end{SaveVerbatim}
\useverb{updwith}

\subsubsection{Labelling effects}

If we have an effectful program \texttt{p} with a single effect, we can 
\remph{label} that effect using the \texttt{(:-)} operator:

\begin{SaveVerbatim}{lblintro}

(:-)  : (l : ty) -> 
        EffM m [x] [y] t -> 
        EffM m [l ::: x] [l ::: y] t

\end{SaveVerbatim}
\useverb{lblintro}

[Example here]

\subsubsection{Introducing effects}

We can introduce a new effect in the course of an effectful program, provided
that the effect can be handled in the current computation context \texttt{m}:

\begin{SaveVerbatim}{neweff}

new : Handler e m => res -> 
      EffM m (MkEff res e :: xs) 
             (MkEff res' e :: xs') a ->
      EffM m xs xs' a

\end{SaveVerbatim}
\useverb{neweff}

\noindent
Once the subprogram is complete, the resource for the new effect is discarded,
as is clear from the type of \texttt{new}.

[Example here]

\subsubsection{Handling failure}

Finally, if the computation context \texttt{m} supports failure handling,
we can use the \texttt{catch} construct to handle errors:

\begin{SaveVerbatim}{catcheff}

catch : Catchable m err =>
        EffM m xs xs' a -> (err -> EffM m xs xs' a) ->
        EffM m xs xs' a

\end{SaveVerbatim}
\useverb{catcheff}


\subsection{The \Eff{} interpreter}

\begin{SaveVerbatim}{effdslinterp}
eff : Env m xs -> EffM m xs xs' a -> (Env m xs' -> a -> m b) -> m b
eff env (value x) k = k env x
eff env (prog `ebind` c) k = eff env prog (\env', p' => eff env' (c p') k)
eff env (effect prf effP) k = execEff env prf effP k
eff env (lift prf effP) k = let env' = dropEnv env prf in 
                                eff env' effP (\envk, p' => k (rebuildEnv envk prf env) p')
eff env (new r prog) k = let env' = r :: env in 
                             eff env' prog (\ (v :: envk), p' => k envk p')
eff env (catch prog handler) k = catch (eff env prog k)
                                       (\e => eff env (handler e) k)
eff {xs = [l ::: x]} env (l :- prog) k = let env' = unlabel {l} env in
                                             eff env' prog (\envk, p' => k (relabel l envk) p')
\end{SaveVerbatim}

\begin{figure*}[t]
\begin{center}
\useverb{effdslinterp}
\end{center}
\caption{The \Eff{} DSL interpreter}
\label{effdsltype}
\end{figure*}


\begin{SaveVerbatim}{resenv}

data Env  : (m : Type -> Type) -> 
            List EFFECT -> Type where
     Nil  : Env m Nil
     (::) : Handler eff m => 
            a -> Env m xs -> Env m (MkEff a eff :: xs)

\end{SaveVerbatim}
\label{sect:envdef}
\useverb{resenv}

\begin{SaveVerbatim}{invokehandler}

execEff : Env m xs -> (p : EffElem e res xs) -> 
          (eff : e res b a) ->
          (Env m (updateResTy xs p eff) -> a -> m t) -> 
          m t
execEff (val :: env) Here eff' k 
    = handle val eff' (\res, v => k (res :: env) v)
execEff (val :: env) (There p) eff k 
    = execEff env p eff (\env', v => k (val :: env') v)

\end{SaveVerbatim}
\useverb{invokehandler}

\begin{SaveVerbatim}{cpsinterp}

eff : Env m xs -> EffM m xs xs' a -> 
      (Env m xs' -> a -> m b) -> m b

\end{SaveVerbatim}
\useverb{cpsinterp}

\begin{SaveVerbatim}{runeff}

run : Applicative m => 
      Env m xs -> EffM m xs xs' a -> m a
run env prog = eff env prog (\env, r => pure r)

runPure : Env id xs -> EffM id xs xs' a -> a
runPure env prog = eff env prog (\env, r => r)

\end{SaveVerbatim}
\useverb{runeff}



\section{Example: An Imperative Language Interpreter}

\newcommand{\Imp}{\texttt{Imp}}

\label{sect:impint}

As a larger example, in this section I present an interpreter for 
a small imperative language, \Imp{}. This language supports variables
with assignment (so requires an environment), introduction of new
variables (so requires the environment to be updatable), 
and printing (so requires console I/O).
We follow the well-typed interpreter pattern again, using a context membership
proof to guarantee that local variables are well-scoped. 

We separate expressions (\texttt{Expr}) from statements in the imperative
language (\texttt{Imp}). First, we define the types. We will support integers,
booleans, and the unit type:

\begin{SaveVerbatim}{imptypes}

data Ty = TyInt | TyBool | TyUnit 

interpTy : Ty -> Type
interpTy TyInt  = Int
interpTy TyBool = Bool
interpTy TyUnit = ()

\end{SaveVerbatim}
\useverb{imptypes}

\noindent
Expressions include values, variables, and binary operators derived from 
\Idris{} functions, and are defined as follows, indexed by a context
\texttt{G} (of type \texttt{Vect Ty n}, a vector of \texttt{n} types),
and the type of the expression:

\begin{SaveVerbatim}{impexpr}

data Expr : Vect Ty n -> Ty -> Type where
     Val : interpTy a -> Expr G a
     Var : HasType i G t -> Expr G t
     Op  : (interpTy a -> interpTy b -> interpTy c) ->
            Expr G a -> Expr G b -> Expr G c

\end{SaveVerbatim}
\useverb{impexpr}

\noindent
For brevity, we omit the definition of \texttt{HasType}. It is sufficient to
know that \texttt{HasType i G t} states that variable \texttt{i} (a de Bruijn
index) in context \texttt{G} has type \texttt{t}.
Values of variables are stored in a heterogeneous list
corresponding to a vector of their types, with a \texttt{lookup} function
to retrieve these values:

\begin{SaveVerbatim}{impenv}

data Vars : Vect Ty n -> Type where
     Nil  : Vars Nil
     (::) : interpTy a -> Vars G -> Vars (a :: G)

lookup : HasType i G t -> Vars G -> interpTy t

\end{SaveVerbatim}
\useverb{impenv}

\noindent
We can write an evaluator for this simple expression language as an 
effectful programming, assuming we have an environment corresponding to
the context \texttt{G}:

\begin{SaveVerbatim}{evalimp}

eval : Expr G t -> Eff m [STATE (Vars G)] (interpTy t)
eval (Val x) = return x
eval (Var i) = do vars <- get
                  return (lookup i vars) 
eval (Op op x y) = [| op (eval x) (eval y) |]

\end{SaveVerbatim}
\useverb{evalimp}

\noindent
Note that, using dependent types, we have expressed a correspondence between
the context \texttt{G} under which the expression is defined, and under which
the variables are defined.

% omit for brevity:
%     For    : Imp G i -> -- initialise
%              Imp G TyBool -> -- test
%              Imp G x -> -- increment
%              Imp G t -> -- body
%              Imp G TyUnit

The imperative fragment is also indexed over a context \texttt{G} and the
type of a program. We use the unit type for statements (specifically
\texttt{Print}) which do not have a value:

\begin{SaveVerbatim}{impprog}

data Imp    : Vect Ty n -> Ty -> Type where
     Let    : Expr G t -> Imp (t :: G) u -> Imp G u
     (:=)   : HasType i G t -> Expr G t -> Imp G t
     Print  : Expr G TyInt -> Imp G TyUnit

     (>>=)  : Imp G a -> 
              (interpTy a -> Imp G b) -> Imp G b 
     return : Expr G t -> Imp G t

\end{SaveVerbatim}
\useverb{impprog}

\noindent
Interpreting the imperative fragment requires the local variables to be
stored as part of the state, as well as console I/O, for interpreting
\texttt{Print}. We express this with the following type:

\begin{SaveVerbatim}{interpimp}

interp : Imp G t -> 
         Eff IO [STDIO, STATE (Vars G)] (interpTy t)

\end{SaveVerbatim}
\useverb{interpimp}

In order to interpret \texttt{Let}, which introduces a new variable with
a given expression as its initial value, we must update the environment.
Before evaluating the scope of the \texttt{Let} binding, the environment
must be extended with an additional value, otherwise the recursive call
will be ill-typed --- the state effect must be carrying an environment
of the correct length and types. Therefore, we evaluate the expression
with \texttt{eval}, extend the environment with the result, evaluate the
scope, then drop the value from the environment.

\begin{SaveVerbatim}{interplet}

interp (Let e sc) 
     = do e' <- eval e
          vars <- get
          putM (e' :: vars)
          res <- interp sc
          (_ :: vars') <- get
          putM vars'
          return res
\end{SaveVerbatim}
\useverb{interplet}

\noindent
Calling \texttt{eval} is fine here, because it uses a smaller set of effects
than \texttt{interp}. Also, not that if we forget to drop the value before
returning, this definition will be ill-typed because the type of
\texttt{interp} requires that the environment is unchanged on exit.

Interpreting an assignment simply involves evaluating the expression to be
stored in the variable, then updating the state, where \texttt{updateVar v vars
val'}
updates the variable at position \texttt{v} in the environment \texttt{vars} with
the new value \texttt{val'}:

\begin{SaveVerbatim}{interpassign}

interp (v := val) 
     = do val' <- eval val
          update (\vars => updateVar v vars val')
          return val'

\end{SaveVerbatim}
\useverb{interpassign}

\noindent
For \texttt{Print}, we simply evaluate the expression and display it, relying
on the \texttt{STDIO} effect:

\begin{SaveVerbatim}{interpprint}

interp (Print x) 
     = do e <- eval x
          putStrLn (show e)

\end{SaveVerbatim}
\useverb{interpprint}

% \begin{SaveVerbatim}{forloop}
% 
% interp (For init test inc body)
%      = do interp init; forLoop 
%   where forLoop = do tval <- interp test
%                      if (not tval) 
%                         then return ()
%                         else do interp body
%                                 interp inc
%                                 forLoop 
% 
% \end{SaveVerbatim}
% \useverb{forloop}
% 
% 
% Remaining operations, Print, return and bind, are straightforward.
% Using DSL notation~\cite{Brady2012} and implicit syntax, we can even make
% the language look like a real imperative language:
% 
% \begin{SaveVerbatim}{countprog}
% 
% dsl imp
%     let = Let
%     variable = id
%     index_first = stop
%     index_next = pop
% 
% implicit MkImp : Expr G t -> Imp G t
% MkImp = Return
% 
% count : Imp [] TyUnit
% count = imp (do let x = 0
%                 For (x := 0) (x < 10) (x := x + 1)
%                     (Print (x + 1)))
% 
% \end{SaveVerbatim}
% \useverb{countprog}

\noindent
Given some program, \texttt{prog : Imp [] TyUnit}, a main program would need
to set up the initial resources, then call the interpreter directly:

\begin{SaveVerbatim}{mainprog}

main : IO ()
main = run [(), []] (interp prog)

\end{SaveVerbatim}
\useverb{mainprog}

\noindent
Though small, this example illustrates the design of a complete application
built using \Eff{}: a main program which sets up the required set of resources
and invokes the top level effectful program. This, in turn, invokes effectful
programs as necessary, which may use at most the resources available at the
point where they are invoked.



\section{Related Work}

\label{sect:related}

The work presented in this paper arose from dissatisfaction with the 
lack of composability of monad transformers, and a belief that a dependently
typed language ought to handle side-effects more flexibly, in such a way
that it would be possible to reason about side-effecting programs. The
inspiration for using an \remph{algebraic} representation of effects was
Bauer and Pretnar's \remph{Eff} language~\cite{Bauer}, a language
based around handlers of algebraic effects. In our \Eff{} system, we have
found type classes to be a natural way of implementing effect handlers,
particular because they allow different effect handlers to be run in
different contexts. Other languages aim to bring effects into their
type system, such as Disciple~\cite{Lippmeier2009}, 
Frank\footnote{\url{https://personal.cis.strath.ac.uk/conor.mcbride/pub/Frank/}}
and
Koka\footnote{\url{http://research.microsoft.com/en-us/projects/koka/}}.
These languages are built on well-studied theoretical
foundations~\cite{Hyland2006,Levy2001,Plotkin2009,Pretnar2010}, 
which we have also applied in this paper.
%
In our approach, we have seen that the \Idris{} type system can express
effects by embedding, without any need to extend the language or type system. 
We make no attempt to \remph{infer} effect types, however, since our intention
is to \remph{specify} effects and \remph{check} that they are used correctly.
Nevertheless, since \texttt{EffM} is represented as an algebraic data type,
some effect inference would be possible.

%Also~\cite{Kammar2012}.

While our original motivation was to avoid the need for monad
transformers~\cite{Liang1995} in order to compose effects, there is a clear
relationship. Indeed, \texttt{EffM}
is in a sense a monad transformer itself, in that it may augment an underlying
monad with additional effects. Therefore, we can expect it to be possible
to combine effects and monad transformers, where necessary. 
The problem with modularity of monad transformers is well-known, and addressed
to some extent~\cite{Jaskelioff2009}, though this still does not allow easy
reordering of transformers, or reducing the transformer stack.
The \Eff{} approach encourages a more fine-grained separation of effects,
by making it easy to call functions which use a smaller set of effects.

Our approach, associating resources with each effect,
leads to a natural way of expressing and verifying resource usage protocols,
by updating the resource type. This is a problem previously tackled by
others, using special purpose type systems~\cite{Walker2000} 
or Haskell extensions~\cite{McBride2011}, and in my own previous
work~\cite{Brady2010a,bradyresource} by creating a DSL for resource management,
but these are less flexible than the present approach in that
combining resources is difficult. 
A related concept is
Typestate~\cite{Aldrich2009,Strom1986}, which similarly allows states to
be tracked in types, though again, we are able to implement this directly
rather than by extending the language. 

To some extent, we can now support imperative programming with dependent
types, such as supported by Xanadu~\cite{Xi2000} and Ynot~\cite{Ynot2008}. 
Ynot in particular is an axiomatic extension to Coq which allows reasoning
about imperative programs using Hoare Type Theory~\cite{HoareTT2008} ---
preconditions and postconditions on operations can be expressed in \Eff{} by
giving appropriate input and output resource types.
%
The difficulty in imperative programming with dependent types is that
updating one value may affect the type of another, though in our interpreter
example in Section \ref{sect:impint} we have safely used a dependent type
in a mutable state.

%(Point about typestate and effects: there are languages built around these
%concepts.  We have done neither, but built a language flexible enough to
%capture the same concepts.)

\section{Conclusions}

\label{sect:conclusion}

\Idris{} is a new language, with a full dependent type system. 
This gives us an ideal opportunity to revisit old problems
about how to handle effects in a pure functional language --- while the old
approach, based on monads and transformers, has proved successful in
Haskell, in this paper we have investigated an alternative approach 
found it to be a natural approach to defining and using effects.
By linking each effect with a \remph{resource} we can even track the state
of resources through program execution, and reason about resource usage
protocols such as file management. 

The \Eff{} approach has a number of strengths, and some weaknesses which
we hope to address in future work. The main strength is that many
common effects such as state, I/O and exceptions can be combined without
the need for monad transformers. Effects can be implemented independently,
and combined without furhter effort. Lifting is automatic --- sub-programs
using a smaller set of effects can be called without any explicit lifting
operation, so as well as being easy to combine effects, programs
remain readable. We have described a number of effects which are representable
in this setting, and there are several others we have not described which
are easy to define, such as parsing, logging, and bounded mutable arrays.
Arrays, statically bounded with a dependent type, could even lead to optimisations
since the effect system can guarantee that only one copy is needed, therefore
the array could have a compact representation with no run-time bounds
checking.

Another advantage is that it will be possible to have fine-grained separation
of systems effects --- we can be precise about needing network support, CGI,
graphics, the operating system's environment, etc, rather than including all of
these effects in one \texttt{IO} monad.

While using \Eff{}, we are still able to use monads.  Effects work \remph{with}
monads rather than against them, and indeed effectful programs are generally
translated to monadic
programs. As a result, concepts which need to be monadic (for example,
continuations) can remain so, and still work with \Eff{}.

\subsection*{Further Work}

The \Eff{} system is promising as a general approach to effectful programming,
but we have only just begun. At present, we are developing more libraries
using \Eff{} to assess how well it works in practice as a progamming tool,
as well as how efficient it is for realistic programs.

The most obvious weakness of \Eff{}, which is already known for the algebraic
approach, is that algebraic effects cannot capture all monads. This does not
appear to be a serious problem, however, given that \Eff{} is
designed to interact with monads, rather than to replace them. More seriously,
but less obviously, there is a small interpreter overhead since \texttt{EffM}
is represented as an algebraic data type, with an associated interpreter.  We
have not yet investigated in depth how to deal with this overhead, or indeed if
it is a serious problem in practice, but we expect that partial
evaluation~\cite{Brady2010} or a finally tagless approach~\cite{Carette2009}
would be sufficient. 

Another weakness which we hope to address is that mixing
control and resource effects is a challenge. For example, we cannot
currently thread state through all branches of a non-deterministic search.
If we can address this, it may be possible to represent more sophisticated
effects such as co-operative multithreading, or even partiality.
One possible way to tackle this problem would be to introduce a new method
of the \texttt{Handler} class, with a default implementation calling the
usual \texttt{handle}, which manages resources more precisely.

An implementation detail which could be improved without affect usage of
the library is that effectful sub-programs require ordering of effects
to be preserved. This should be a small improvement, merely requiring permuation
proofs of lists, but ideally generating these proofs will be automatic.

The \Eff{} implementation is entirely within the \Idris{} language --- no
extensions were needed. It takes advantage of dependent types and theorem
proving in several small but important ways: heterogenous lists of resources,
proofs of list membership and sub-list predicates, and parameterisation of
resources. Since modern Haskell supports many of these features, this leads
to an obvious question: what would it take to implement \Eff{} as an embedded
library in Haskell? An interesting topic for further study would be whether
this approach to combining effects would be feasible in a more mainstream
functional language.

Monad transformers have served us well, and are a good fit for the Haskell
type system. However, type systems are moving on. Perhaps now is the time to
look for something more effective.



%\section*{Acknowledgments}

%This work was partly funded by the Scottish Informatics and Computer
%Science Alliance (SICSA), by EU Framework 7 Project
%No. 248828 (ADVANCE) and by EPSRC grant EP/F030657 (Islay).
%We thank our colleagues, notably William Cook, James McKinna and Anil Madhavapeddy for several helpful
%discussions, and the anonymous reviewers for their constructive suggestions on this paper.

\bibliographystyle{abbrv}
\begin{small}
\bibliography{library}

\appendix
%\input{code}

\end{small}
\end{document}
